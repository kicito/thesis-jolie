\chapter*{Introduction}

% introduction 
% ความเป็นมาและความสำคัญของปัญหา,จุดประสงค์,ขอบเขต,benefit ที่จะได้จากโปรเจคนี้

%idea 2 ---------------------
% ทำไมถึงมี SOC / Modular เพราะsoftware มีความเปน module?
% มีแล้วช่วยให้ software development ดีขึ้นยังไงลงดีเทลหน่อยนึง
% ด้วยประโยชน์ของภาษาที่รองรับ modular programmingจะช่วยให้การพัฒนาsoftware ดีขึ้นยังไง ยกตัวอย่าง
% ดังนั้นนี่จึงเป็นที่มาของ โปรเจคนี้โดยมีวัตถุประสงดังนี้ 
% บอกจุดประสงค์ ---> ศึกษาส่วนประกอบของภาษาโปรแกรมมิ่งที่รองรับ module system และพัฒนา Module system สำหรับภาษา Jolie 
% บอกขอบเขตของโปรเจคที่เราจะทำ(ของพี่แดนเขียนขอบเขตไว้ใน stucture of this thesis
% ของเราก็เขียนว่า feature อะไรบ้างที่เพิ่มเข้าไป
% ปิดท้ายด้วย benefit ที่จะได้จากการศึกษาและเพิ่ม feature ของ jolie

% What - why - how
% ทำไมถึงมี SOC / Modular เพราะsoftware มีความเปน module?


% เขียนแนวว่า ปัจจุบันอุสาหกรรม software มีความใหญ่และซับซ้อนขึ้น ได้มีการศึกษาถึงวิธีและ practical guideline and methodology to create a หนึ่งในวิธีการที่จะทำให้โค้ด maintain ได้ง่ายขึ้นคือการ ทำให้ system มีความเป็น modular โดยการ apply Seperation of concerns to plan and write code, 

For a long time, The demand for the principle and methodology of designing software has always been a point of discussion in software development.
As the software continuously evolved from time to time, Israeli et al. studied Linux kernel evolution for over 800 new versions and found that the size, from various metrics, has been increasing linearly from 1994 to 2008 \cite{10.1016/j.jss.2009.09.042}. Even for the present day, the development of the Linux kernel between 2019 to 2020, 1.3 million new lines of codes have been added to Linux kernel source code by 4,190 different contributors \cite{anderson2020}. For this reason, the goal of having straightforward, maintainable details in software, from an architectural point of view to the implementation level is crucial for every software development cycle.

The principle that is well known and adopted in many aspects of the related programming field is the Separation of Concerns (\textbf{SoC}). SoC aims to separate the problem into a small set of sub-problems to help the practitioner focus on one aspect, instead of being overwhelmed by the development complexity as a whole. Among many other principles, modularizing the system to smaller components is one of the prominent practices widely used in the industry.

\begin{displayquote}
    "the separation of concerns", which, even if not perfectly possible, is yet the only available technique for effective ordering of one's thoughts, that I know of. This is what I mean by "focussing one's attention upon some aspect"
    - Edsger W.Dijkstra\cite{dijkstra1974}
\end{displayquote}


The principle that is well known and adopted in many aspects of the related programming field is the Separation of Concerns (\textbf{SoC}). SoC aims to separate the problem into a small set of sub-problems to help the practitioner focus on one aspect, instead of being overwhelmed by the development complexity as a whole. Among many other principles, modularizing the system to smaller components is one of the prominent practices widely used in the industry.

Many software designs and principles adopt SoC as its fundamental principle to pursue an easy to maintain, interchangeable, and reusable software. From the architectural design point of view to the implementation detail aspect. From the very top level, Service-oriented architecture (\textbf{SOA}) focuses on the composition of services via communication over a network protocol. At the implementation level of software, modular programming is a design technique where the software is divided into several small, self-contain units, so-called a module. 

% // Service in SOA and Microservice
The SOA concept is well known for more than a decade, as it helped reduce the integration cost of information between services by providing a standard of data format. However, the developers and designers suffer from the complexity of implementing an SOA; one of the difficulties is the standard definition of the compositional protocol between services. Consequently, adopting the SOA concept has a trade-off between the integration cost as a business value and the additional cost of development. As we see, the beneficial organization for SOA, such as IT-related large companies or the business association, invested and experimented with the concept and published the definitions on their own\cite{cesar2020, ibmcloudeducation2019}.
% //https://opentravel.org/about-us/ 

The SOA system's complexity caused it to suffer from gaining technical maturity; thus, the attempt to rebrand SOA for the real-world application has resulted in the Microservice Architecture, which considered as an the second iteration of SOA \cite{Dragoni2017}. Microservice Architecture is a widespread software architectural paradigm whereby a service is defined as a standalone service, which means that it may not be composed of a combination of services. This definition reduces the complexity of implementing as it forces the designer to encapsulate only single business logic into a service, which makes the Microservice easier to implement and become a prominent software design at present. In recent years, it has become state-of-the-art software development, adopted by companies such as Amazon, Google, and Facebook.

Jolie\cite{JOLIE} is a service-oriented programming language that addresses the difficulty of developing a distributed system of services. Jolie tackles the difficulty of service implementation from the linguistic approach, unlike other general programming languages that usually require a combination of external technology building on top of each other to define a service. Jolie contributes a seamless experience for the developer in developing service as it provides the capability to define both communication method details and the workflow for the invocation operation.

The goal of this thesis is to develop a foundation of the module system for Jolie programming languages. We explore the design choices module system a develop one that is best-suited for Jolie's. We also have an experimental implementation of enhancement in Jolie type system to allow Jolie developer to define value constraints for a type. It is inspired by the work defining a schema for the JavaScript Object Notation (JSON)\footnote{JSON Schema https://json-schema.org/}, which is a well-known
data format used for communication between services in the present day.

The current work progress for Jolie Module System is partially merged into the Jolie the main development \url{https://github.com/jolie/jolie} repository is available at \url{https://github.com/kicito/jolie}.

% \section*{Structure of thesis}

This thesis structure as following
\begin{itemize}
    \item Chapter 1: discussion of the Module System and brief introduction of Jolie Programming Language
    \item Chapter 2: discussion of Jolie's Module system design space
    \item Chapter 3: description of Jolie's Module system implementation details
    \item Chapter 4: conclusion of this thesis
\end{itemize}
 
% % \section*{Notation used}

% \subsection*{Grammar}

% For this document, we use Extended Backus–Naur form (EBNF) notation to express the grammar of the programming language syntax. The following figure shows the example of grammar and the common syntax that will be used in the rest of this book 
% \begin{figure}[h!]
%     \begin{framed}
%         \begin{grammar}
%             <jolie> ::= <definitons>* <deployments>*  `main' `{' <behaviors> `}'

%             <definitons> ::= <typeDef> \alt \dots \alt <interfaceDef> 

%             <deployments> ::= \dots

%             <behaviors> ::= \dots
%         \end{grammar}
%     \end{framed}
%     \caption{An example grammar for Jolie}
% \end{figure}




% SOC เพื่อ ปูไปยัง Jolie
% Service-Oriented Computing (\textbf{SOC}) is a computing paradigm for developing an application by composing different services.
% Each of service defines it's own role and functionality in the application context\cite{papazoglou2003service}. 
% A service, or so called a program, proceeds it's execution by passing message from one to another. 
% The interest on \textbf{SOC} concept has been continuously gain attraction since \cite{JOLIE}.
% One of the research field related to \textbf{SOC} is the study of concurrency theory behind a process flow language, from it's specification of process by passing message, it can be formalizing the existing work flow language into message passing process calculus such as \(\pi\)-calculus.
% Similarly in industrial field, despite of missing the usage of term \textbf{SOC}, is focusing on the standardize the structured message; It is emphasized on constructing a seamless integration between service providers.

% Developing a service-oriented software had been criticized for being too constrained by the application it serve \cite{mckendrick_2009} \cite{SOA_opengroup}.
% This leads to an attempt to realizing the concept in the current state and implementation approach for \textbf{SOC}.
% The successful attempt of defining a modern service-oriented approach for develop a system is Microservice, where many of constrains has been relaxed in order to make implementation feasible for . 

% By using a set of expressive service based primitives. Jolie is also defined as a programming language for develop microservices as a code \cite{DBLP:journals/corr/GuidiLMM17}.  Jolie is 
% Supported modulesystem programming language

% intro to microservice
% Dragoni et al. studied and identified the history of the service-oriented architectural design and classified Microservice as a second iteration of an software design architecture that adopting \textbf{SOC} concept.
% Microservice has been exponentially gained it's popularity from since the term first coined in 2005 by Dr. Peter Rodgers, one of the major reason of Microservice success is due to the availability of tools and technology, this let development team in a company adopts the design pattern easily.

% SOA
% From the very top level, a software design style that embracing the modular design is called service-oriented architecture (\textbf{SOA}). \textbf{SOA} is a software architecture design that is focusing on the composition of services via a communication over a network protocol. The majority of \textbf{SOA} is implemented with Web Services technology, where services are composed together with the internet protocol such as SOAP, or HTTP. The implementation of \textbf{SOA} has been suffered by many difficulties mostly due to the following: lacking of tooling and testing framework \cite{vithanage_2014} \cite{JPM} and standard definition on the composition protocol between services. The latter reason had led to publication of several company owned definitions \cite{cesar2020} \cite{ibmcloudeducation2019}. \textbf{SOA} has contribute to a business value in many industry, it provides a standard of specification to ensure a seamless experience for connecting multiple systems together, for example, OpenTravel Alliance is an association by travel companies, which is focusing on develop a message structure for communication between global travel agency. It can be seen that majority of system that has adopt \textbf{SOA} is mostly a group of existing companies which shared the enterprise business context. From these reason has led to the difficulty on for a software architect to design a proper \textbf{SOA} style in a small system, and resulted in a problematic software with high technical dept.





% The early stage of \textbf{SOA} was focusing on defining a document format, in order to allow a service to be reuseable by different web service provider or

% maybe to background
% In \textbf{SOA}, A service is consist of four properties which are \cite{SOA_opengroup} :

% \begin{itemize}
%     \item a logical representation of a repeatable business activity that has a specified outcome
%     \item self-contained
%     \item may be composed of other services
%     \item a “black box” to consumers of the service
% \end{itemize}

% \textbf{SOA} is being reknown from 2005 from the rising of World-Wide web, where the emerging of web services has start to claim their interest from industry aspects. The \textbf{SOA} during those time was about working on definition of documents to be communicate with others to be able to query. this 

% The \textbf{SOA} had not initially got well received, due to it's complexity of the implementation and challenging of developing a testing framework \cite{vithanage_2014}.

% The top level of designing a software that applying this fundamental is called
% % collection of components defines a service, which usually responsible for a certain context in a business unit,


% With modular system, the program is analyzed and build up from bottom point of view; separating the concerns that has to be address in order to achieve the end product.

% Separation of concern

% It also helps reducing complexity on designing a system.

% Once it finished and get build up by composing as a whole.

% Separating of concerns has been develop in many aspects of development cycle, from the high level architectural design such as

% % what Module system for jolie
% Modular system, a system that built from composing a small unit so called module, has becoming a well adopted principle for

% Separation of concern is a

% A Modular program is a program that

% Sepa



% Modular programming is one of the key




% Jolie คืออะไร (ที่มา)
% % Microservice
% % เริ่มจาก SoC -> SOCK -> Jolie
% % มาจาก process calculi SOCK
% % Microservice is a programming practice that focusing on decomposing an application into small components which responsible for a single functionality for
% % 
% %  SOC
% Service-Oriented Computing (\textbf{SOC}) is a computing paradigm for developing an application by composing different services. Each of service defines it's own functionality and role in the application context\cite{papazoglou2003service}. A service, or so called a program, proceeds it's execution by passing message from one to another. The interest on \textbf{SOC} concept has been continuously gain attraction since \cite{JOLIE}. One of the research field related to \textbf{SOC} is the study of concurrency theory behind \textbf{SOC}'s workflow language, from it's specification of process by passing message, it can be formalizing the existing work flow language into message passing process calculus such as \(\pi\)-calculus. Similarly in industrial field, despite of missing the usage of term \textbf{SOC}, Dragoni et al. studied and identified Microservice as a second iteration of an software design architecture that adopting \textbf{SOC} concept. Microservice has been exponentially gained it's popularity from since the term first coined in 2005 by Dr. Peter Rodgers, one of the major reason of Microservice success is due to the availability of tools and technology, this let development team in a company adopts the design pattern easily.

% % SoA patterns chroreography and orchestration
% Approaches for designing a service-oriented application can be classified, by composition of services, in two patterns. The definitions of these two patterns have been developed from observing arts performance. The first approach, \textbf{choreography}, is the pattern where composition of service is done by considering as each of services perform their execution on it's own context, exposing an interfaces and it's location for others to connect. A service in choreography
% % A simple metaphor for describe this approach is considering a dancer who is performing their dance on a stage, without any leader, they perform their art by their self, only make a contract to other when the timing is right.
% On the other hand, \textbf{orchestration} is a pattern of centralizing the service's behavior by having a single service so call an orchestrator, coordinate and conduct a group of service to perform a task.

% % Jolie





% The concept of separating the concerns design principle has been mentioned on a published paper from   , the Service-oriented architecture (\textbf{SOA})


% % Modular Programming คืออะไร (ที่มา)
% % มีรูปแบบยังไง
% % ประโยชน์
% % ภาษาอะไรมีบ้าง
% % Module system
% Modular programming is one of the key component in modern programming languages. The emphasis of modularity is to separate functionality or concerns of a program to be an independent module such that it is self sufficient; it contains every information needed to perform a specific task.




% % Scope working importstatement for Jolie, alternate file Layout for jolie execution file