\chapter*{Introduction}
The need for principles and a methodology of designing software has always been a point of discussion in software development. The striving for better software quality is the goal of software engineering, a discipline in computer science. Bauer \cite{DBLP:conf/ifip/Bauer71} first coined the term ‘software engineering’ with the definition as \say{the establishment and use of sound engineering principles to obtain economically software that is reliable and works on real machines efficiently}. The study initiated from the lack of the method of writing an efficient program in the early days of computer science, as the computing power of computers started to outpace the capability of programmers to utilize those capabilities effectively.

As the software continuously evolved from time to time, Israeli et al. studied Linux kernel evolution for over 800 new versions and found that the size, from various metrics, has been increasing linearly from 1994 to 2008 \cite{10.1016/j.jss.2009.09.042}. Even for the present day, the development of the Linux kernel between 2019 to 2020, 1.3 million new lines of codes have been added to Linux kernel source code by 4,190 different contributors \cite{anderson2020}. For this reason, the goal of having straightforward, maintainable details in software, from an architectural point of view to the implementation level is crucial for every software development cycle.

The principle that is well known and adopted in many aspects of the related programming field is the Separation of Concerns (\textbf{SoC}). SoC aims to separate the problem into a small set of sub-problems to help the practitioner focus on one aspect, instead of being overwhelmed by the development complexity as a whole. Among many other principles, modularizing the system to smaller components is one of the prominent practices widely used in the industry.

Many software designs and principles adopt SoC as its fundamental principle to pursue an easy to maintain, interchangeable, and reusable software. For example, the design of internet protocol we use nowadays consists of a stack of interchangeable technology which allows the developer to focus on the function of the specific layer and neglect the others\cite{stack-1994}. The Model-View-Controller (MVC) design pattern, which is a well-received pattern for web application development nowadays, is highly influenced by the SoC principle as it separates the program logic into three components as indicated by its name. The MVC pattern pursues the modularity for the interactive application and gives the developer a clear perspective on the application components and reuses the developed component in the new application \cite{GlennACF}.

One paradigm that applies SoC for an architectural style of a system is  Service-oriented architecture (\textbf{SOA}). The SOA focuses on the composition of services via communication over a network protocol. A service usually represents an enterprise business process. The SOA approach allows the business operation to develop its service and interacts with other services via message exchanging. The SoA system benefits from its modularity on the functional and service reusability\cite{SOAopengroup}.

% SOA and microservice
The SOA concept has started to gain recognition for almost two decades, as it helps reduce the integration cost of information between services by providing a standard of data format. However, the developers and designers suffer from the complexity of implementing an SOA. For example, the architecture of Hewlett-Packard's internal service is reported to be too complicated \cite{davenport-2016}. Moreover, a request handling from a service could involve an interaction with multiple services, which raises trust issues, especially if it is performed by a third-party service provider.

Since the emergence of the SoA, one of the successful SoA applications that provide business value are those used by particular industries to standardize document message information between services. This formatted set of information significantly reduces the integration cost for different services within the same line of business e.g. the exchange of information between online travel agencies\footnote{
    OpenTravel Alliance: https://opentravel.org/about-us/
} and hotels 
\footnote{
    Hospitality Technology Next Generation: https://www.htng.org/
}.

The SOA system's complexity caused it to suffer from gaining technical maturity; thus, the attempt to rebrand SOA for real-world application has resulted in the Microservice Architecture. According to Dragoni et al., microservice architecture is considered the second iteration of SOA. Microservice architecture is a distributed application where all modules are microservices\cite{Dragoni2017}. Microservice architecture aims to reduce implementation complexity as it forces the designer to encapsulate the business logic into a composition of simple services. This makes the Microservice easier to implement and it became a prominent software design pattern. In recent years, it has become the state-of-the-art software development, adopted by companies such as Amazon \cite{iii-2015}, Netflix\cite{blog-2018}, and Spotify\cite{cope-2020}.

Jolie\cite{JOLIE} is a service-oriented programming language that addresses the difficulty of developing a distributed system of services. Jolie tackles the difficulty of service implementation from the linguistic approach. Rather than building up the program from objects or functions, Jolie code is contained in a service \cite{jolie-website}. Jolie contributes a seamless experience in developing microservice applications by providing a set of primitives to define both communication details and workflow for the language's invocation operation.

At the implementation level of software, modular programming is a design technique that applies SoC principle. The modular software concept divides an application functionality into several small, self-contained units, so-called a module. A modular program usually benefits the developer compared to monolithic application in many ways \cite{mall-2018}. As the size of modules is relatively small and encapsulates a single functionality, which helps the developer to focus and contribute to software quality. It also adds efficiency to the development process as the development could be achieved through collaboration between different development teams, who are responsible for different modules to be plugged together through predefined API.

The support of the module concept in a programming language can significantly benefit the modular programming system. The programming language could implement a specific mechanism to handle module composition, so-called a module system. The module system can detect defections before the execution, such as module location resolution, and compatibility checking between symbols.

This thesis aims to develop a foundation of the module system for Jolie microservices, using Jolie programming language as a reference implementation technology. We explore the design choices of such a module system and develop one that is best-suited for Jolie. Our development led us to identify an additional companion feature for the module system: refinement type, which supports the type-level definition of constraints on data. This feature becomes useful to module developers in providing safer modules/ libraries to clients. Thus, we also developed an experimental implementation of enhancement of the Jolie type system to allow Jolie developer to define refinements on types. The latter development is inspired by the work defining a schema for the JavaScript Object Notation (JSON)\footnote{JSON Schema https://json-schema.org/}, which is one of the main data formats used for message passing between services in the present day.

The current work for the Jolie Module System is partially merged into the Jolie the master development \url{https://github.com/jolie/jolie} and the main repository is available at \url{https://github.com/kicito/jolie}.
