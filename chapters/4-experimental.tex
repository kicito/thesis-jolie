\chapter{Experimental Design: Refinement type}

In this chapter, we discuss the experimental feature refinement types. Specifically, we believe that it has potential for the module system to define constrains on the configuration data passed to modules through type checking. The current implementation is not accepted feature to the Jolie main repository, but it is available in the author's private repository\footnote{https://github.com/kicito/jolie/tree/jolie-module-3}.

Tchitchigin et al. studied refinement types for Jolie to implement a dynamic and static type checking for Jolie native types with the type verification done by the SMT solver \cite{DBLP:journals/corr/TchitchiginSMEM16}. The study resulted in the reduction of the testing effort and enhanced the security of the service.

In our implementation, inspired by the JSON Schema\footnote{JSON Schema https://json-schema.org/}, we extend Jolie type system to support the creation of a value format for each field. We also add a capability to define a default value for a service parameter. Whenever the value is missing during type validation, instead of raising an error, the default value is used, and the execution proceeds.

\begin{figure}[ht]
    \begin{framed}
        \begin{grammar}
            <refinedType> ::= `type' <ID> `:' <type>  `(' <refinedTypeRule> ( `,' <refinedTypeRule>)* `)'
            
            <refinedTypeRule> ::= <rulename> `(' ruleArg ( `,' ruleArg )* `)'
            
            <ruleArg> ::= <expression> | <refinedTypeRule>
            
            <typename> ::= ID
            
            <rulename> ::= ID
        \end{grammar}
    \end{framed}
    \caption{Grammar for Jolie's refinement types declaration}
    \label{fig:JolieRefinementGrammar}
\end{figure}

The syntax of refined types is extended from the existing type syntax, as a set of type rules defined within the parenthesis after the type construction. The \(<rulename>\), as shown in figure ~\ref{fig:JolieRefinementGrammar}, is a set of available predicates for the specific type. The table ~\ref{fig:jolie-refinement-rules} describes the available predicate for each native type supported by Jolie.

\begin{figure}[ht]
    \begin{tabular}{ |l|l|l| }
        \hline
        Type    & Predicate                                            & Description                   \\
        \hline
        boolean & default(x)                                           & is_defined(value) ? value : x \\
        \hline
        \multirow{5}{4em}{number}
                & default(x)                                           & is_defined(value) ? value : x \\
                & minimum(x)                                           & value $\ge$ x                 \\
                & exclusiveMinimum(x)                                  & value $>$ x                   \\
                & maximum(x)                                           & value $\le$ x                 \\
                & exclusiveMaximum(x)                                  & value $<$ x                   \\
        \hline
        \multirow{4}{4em}{string}
                & default(x)                                           & is_defined(value) ? value : x \\
                & minLength(x)                                         & value $\ge$ x                 \\
                & maxLength(x)                                         & value $\le$ x                 \\
                & pattern(x) \footnotesize{where x is an regex expression} & x.match(value)                \\
        \hline
    \end{tabular}
    \caption{Jolie's refinement types rules}
    \label{fig:jolie-refinement-rules}
\end{figure}

With this, a simple yet powerful set of predicates available in refinement types, so that the developer is capable of defining a value constraint and control the possible value of the receiving parameter, as shown in the example below of the \textit{twiceService} in figure ~\ref{fig:jolie-refinement-example}.

\begin{listing}[ht]
    \lstset{language=Jolie,
        style=codeStyle,
        numbers=left,
        firstnumber=1
    }
    \begin{lstlisting}[frame=tlrb]{refinement-type-service.ol}
interface twiceIface{
    RequestResponse: twice(int)(int)
}

type twiceParam: void{
    .location: string(
        pattern(^socket:\/\/localhost:3000$)
    )
    .protocol: string(default("sodep"))
}

service twiceService(param : twiceParam){
    inputPort myPort{
        location: param.location
        protocol: param.protocol
        interfaces: twiceIface
    }

    main{
        twice(req)(res){
            res = req * 2
        }
    }
}
\end{lstlisting}
    \caption{Jolie implementation of the twiceService with refinement types}
    \label{fig:jolie-refinement-example}
\end{listing}

Here, we defined \textit{twiceService}, which accepts a value of type \textit{twiceParam}; the received value is later used as a configuration of the incoming communication port \textit{myPort}. The refinement type of \textit{location} and \textit{protocol} are specified in the way that the service developer of \textit{twiceService} can assure that the received value from the client matches the predicate given in the type declaration.

Refinement types are a practical and powerful feature that could be a research topic in the further development of Jolie. As we are seeing, refinement types gives service owner support to determine and limit the possible configuration from the embedder service. Yet, with this specification, it also introduces the uncertainty of execution behavior for Jolie type checking and may yield an unexpected result; for example, given a list of condition predicates and the default rule, what is the proper execution order of predicate checking? Another crucial case is the method to detect conflict between predicates. For these reasons, it is an interesting future research to study the Jolie type refinements system for enhancing the experience of developing Jolie services.
