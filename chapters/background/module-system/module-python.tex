% Content layout for Module system

% history
% definitions
% import stmt syntax
% import mechanism


\subsection{Python Module System}

The Python module system history and it's design can be trace back to Modula-3, as its concept of modularity is influenced many modern programming languages including Python.
Python acquired the Modula-3\cite{modula-3} usage of \texttt{dot} notation for traversing between modules and elements inside it\cite{python-foreward-essay}.

Python’s module system has been gradually evolving as the showing in more than 50 proposals related to module system in Python Enhancement Proposals (PEP) \cite{pep0}. Here in this chapter, we will discuss how Python 3.0 defined their module system and their importing mechanism.

A \textbf{module} is an object with a list of python definitions and executable statements. These statements are executed during the initialization of the module. In the file system perspective a module is file with py extension, and module is named after the file name.

A \textbf{package} is a module which might contain sub-modules or a sub-packages.
Thus, in file system, a directories are recognized as Python packages.
In current version of Python, the packages are categorized into two types called regular packages and namespace packages.
A \textbf{regular package} is a package that is implemented as a directory containing \textbf{\_\_init\_\_.py} file.
The directory without a \textbf{\_\_init\_\_.py} is consider as a newer type package so-called \textbf{namespace package}.
The namespace package was introduced to natively support the declaration of a module from different location \cite{pep420}. The element in namespace package is called a portion which can be a set of files in a single directory or an entry in a zip file.

\subsubsection{Import Statement}

Despite various ways for a module to invoke an import mechanism, we mainly focus on an import statement of Python.
The location of the target module can be specified by a dotted separate module name, while the target symbols is used to indicate symbol name in local namespace to be referred later for import statement caller or a wildcard character.
The syntax of import statements in Python is defined as follows in figure ~\ref{fig:python-import-stmt-syntax}.

% A regular package is a directory containing \_\_init\_\_.py file. This file is executed when the package of this type is imported, and any object defined in \_\_init\_\_.py will be bound to package namespace. The \_\_init\_\_.py file is also used by package developer to control definitions to be import when the local importer specify a wildcard character(*) as an identifier target; this is done by assigning exported symbols to \_\_all\_\_  symbol.

\begin{figure}[ht]
    \begin{framed}
        \begin{grammar}
            <importStmt>
            ::= <importName>
            | <importFrom>

            <importName>
            ::= `import' <dottedAsNames>

            <importFrom>
            ::= `from' ( `.' )* ( <dottedName> | `.' ) `import' ( `*' | <importAsNames> )

            <importAsNames>
            ::= <importAsName> ( `,' <importAsName> )*

            <importAsName>
            ::= <NAME> ( `as' <NAME> )?

            <dottedAsNames>
            ::= <dottedAsName> ( `,' <dottedAsName> )*

            <dottedAsName>
            ::= <dottedName> ( `as' <NAME> )?

            <dottedName>
            ::= <NAME> ( `.' <NAME> )*

            <NAME> ::= <ID>
        \end{grammar}
    \end{framed}
    \caption{Python import statement syntax }
    \label{fig:python-import-stmt-syntax}
\end{figure}

An import statement of python has two forms, it is either in form of \texttt{<importName>} and \texttt{<importFrom>} statements. Both statements create identical instructions to the module importer, choice of use is subjectively open to the developer to pick. The semantic of these to form is differ from each other in the following way.

The first form of import, \texttt{<importName>}, every \texttt{<NAME>} declared in \texttt{<dottedName>} excepted last one must be a package, while the last \texttt{<NAME>} can only be either a module or a package; it cannot referred to a definition symbol inside a module.

For the second form, \texttt{<importFrom>}, the item element can be a module or any definition symbol declared within the package. In this form, it performs a lookup to symbol item inside the package, if not found, the process continues with an attempt to import module item. The table ~\ref{table:python-import-def} shows the translation of symbols identified in different form of import symbol.

\begin{table}[ht]
    \centering
    \begin{tabular}{ |l|p{0.7\linewidth}| }
        \hline
        Statement                   & Element explanation                                                                                                                                                 \\
        \hline
        import item                 & \textit{item} can either be a package or a module.                                                                                                                  \\
        \hline
        import item.subitem         & \textit{item} is a package while \textit{subitem} is a module in \textit{item}.                                                                                     \\
        \hline
        from pkg import item, item2 &
        \makecell[l]
        {If \textit{pkg} is a module,                                                                                                                                                                     \\ \textit{item} and \textit{item2} are definition symbols in \textit{pkg}                                                            \\
            If \textit{pkg} is a package,                                                                                                                                                                 \\ \textit{item} and \textit{item2} can either be a module in \textit{pkg} or \\ a definition defined in \textit{pkg}.
        }                                                                                                                                                                                                 \\
        \hline
        from pkg.mod import item    & \textit{pkg} is a package and mod is a module in \textit{pkg}, while \textit{item} can be a module of \textit{mod}, or a definition symbol defined in \textit{mod}. \\
        \hline
    \end{tabular}
    \caption{Python import element definition}
    \label{table:python-import-def}
\end{table}

There are two types of import statement in Python which determine the module lookup location \cite{pep328}. Absolute import is a default behavior for an import statement, where it will perform lookup based on the list of system module lookup path. Other type of import statement, relative import, instruct system to perform lookup for module relatively from the module that an import statement reside in.

A useful feature that \texttt{<importFrom>} provides to Python developer is an ability to specified a module target location. If the module target is prefixed with a dot(.), it will be considered as a relative import, where a dot serve as go one level up.


\subsubsection{Import Mechanism}

After the execution of import statement is completed, it is resulted in a set of defined definitions ready to use in the caller’s execution context.
In order to achieve this goal, Python processes import statement in two step, during creation of symbol table, which the importing target symbols are created in global execution scope of the import statement caller, then the runtime execution of it is combining searching of the named module and binding result to the importing module's symbol reference.

In the implementation level, this two operations of execution are perform by an abstraction called \texttt{importer}, which consists of two other crucial abstraction objects; finders and loaders.
Finders are responsible for performing module lookup, while loaders use the result from finder to execute and create a module object which later is saved to the cache.
Lastly loader binds name as defined in import statement to the local execution environment.

Python locate a module by a \texttt{finder} which is categorized into two types based on the location where the finder should attempt to find the module.
Firstly the \texttt{MetaPathFinder} will perform a lookup at a list of paths passing through the finding method.
The second type of finder, the \texttt{PathEntryFinder}, performs look up only to the path that is assigned during its creation.
Searching operations may cause certain side-effects to the execution environment. In case of importing a module inside package, it is process by importing each parent packages from top level until the module is reached, which means it also executes the initialization instructions of every module target's parent packages.

After module has been loaded into the execution environment, the importing target symbols are binding to into the importer global execution scope this scope is also called the module scope. There is only a single global scope in python program execution.

\subsubsection{Symbol Privacy}

In python, every declaration of symbol is considered a private symbol if it has a prefix underscore(\_). Such symbol will be ignored for a wildcard import (from ... import *) and error will be raised when it is implicitly referring to, as the declaration can be only used in it's own module scope.
