
\subsection{Python Module System}

The Python module system history and its design history can be traced back to Modula-3, as its modularity concept is influenced by many modern programming languages, including Python.
Python acquired the Modula-3\cite{modula-3} usage of \texttt{dot} notation for traversing between modules and elements inside it\cite{python-foreward-essay}.

Python's module system has been gradually evolving as the showing in more than 50 proposals related to the module system in Python Enhancement Proposals (PEP) \cite{pep0}. This chapter will discuss how Python 3.0 defined its module system and its importing mechanism.

A \textbf{module} is an object with a list of python definitions and executable statements. These statements are executed during the initialization of the module. In the file system perspective, a module is a file with 'py' extension, and the module is named after the file name.

A \textbf{package} is a module which might contain sub-modules or a sub-packages.
Thus, in the file system, directories are recognized as Python packages.
In the current version of Python, the packages are categorized into two types called regular packages and namespace packages.
A \textbf{regular package} is a package that is implemented as a directory containing \textbf{\_\_init\_\_.py} file.
The directory without a \textbf{\_\_init\_\_.py} is consider as a newer type package so-called \textbf{namespace package}.
The namespace package was introduced to natively support the declaration of a module from different locations \cite{pep420}. The element in the namespace package is called a portion, which can be a set of files in a single directory or an entry in a zip file.

\subsubsection{Import Statement}

Despite a few methods for a module to invoke an import mechanism, we mainly focus on Python's import statement.
A dotted separate module name can specify the location of the target module. The target symbols are used to indicate symbol name in local namespace to be referred later for import statement caller or a wildcard character.
The syntax of import statements in Python is defined as follows in figure ~\ref{fig:python-import-stmt-syntax}.

\begin{figure}[ht]
    \begin{framed}
        \begin{grammar}
            <importStmt>
            ::= <importName>
            | <importFrom>

            <importName>
            ::= `import' <dottedAsNames>

            <importFrom>
            ::= `from' ( `.' )* ( <dottedName> | `.' ) `import' ( `*' | <importAsNames> )

            <importAsNames>
            ::= <importAsName> ( `,' <importAsName> )*

            <importAsName>
            ::= <NAME> ( `as' <NAME> )?

            <dottedAsNames>
            ::= <dottedAsName> ( `,' <dottedAsName> )*

            <dottedAsName>
            ::= <dottedName> ( `as' <NAME> )?

            <dottedName>
            ::= <NAME> ( `.' <NAME> )*

            <NAME> ::= <ID>
        \end{grammar}
    \end{framed}
    \caption{Python import statement syntax }
    \label{fig:python-import-stmt-syntax}
\end{figure}

An import statement of python has two forms, it is either in form of \texttt{<importName>} and \texttt{<importFrom>} statements. Both statements create identical instructions to the module importer, choice of use is subjectively open to the developer to pick. The semantic of these to form is different from each other in the following way.

The first form of import, \texttt{<importName>}, every \texttt{<NAME>} declared in \texttt{<dottedName>} excepted the last one must be a package, while the previous \texttt{<NAME>} can only be either a module or a package; it cannot be referred to a definition symbol inside a module.

For the second form, \texttt{<importFrom>}, the item element can be a module or any definition symbol declared within the package. In this form, it performs a lookup to symbol item inside the package; if not found, it continues with an attempt to import module items. The table ~\ref{table:python-import-def} shows the translation of symbols identified in a different form of import symbol.

\begin{table}[ht]
    \centering
    \begin{tabular}{ |l|p{0.7\linewidth}| }
        \hline
        Statement                   & Element explanation                                                                                                                                                 \\
        \hline
        import item                 & \textit{item} can either be a package or a module.                                                                                                                  \\
        \hline
        import item.subitem         & \textit{item} is a package while \textit{subitem} is a module in \textit{item}.                                                                                     \\
        \hline
        from pkg import item, item2 &
        \makecell[l]
        {If \textit{pkg} is a module,                                                                                                                                                                     \\ \textit{item} and \textit{item2} are definition symbols in \textit{pkg}                                                            \\
            If \textit{pkg} is a package,                                                                                                                                                                 \\ \textit{item} and \textit{item2} can either be a module in \textit{pkg} or \\ a definition defined in \textit{pkg}.
        }                                                                                                                                                                                                 \\
        \hline
        from pkg.mod import item    & \textit{pkg} is a package and mod is a module in \textit{pkg}, while \textit{item} can be a module of \textit{mod}, or a definition symbol defined in \textit{mod}. \\
        \hline
    \end{tabular}
    \caption{Python import element definition}
    \label{table:python-import-def}
\end{table}

There are two types of import statement in Python which determine the module lookup location \cite{pep328}. Absolute import is the default behavior for an import statement, where it will perform a lookup based on the list of system module lookup path. Other types of import statements, relative import, instruct the system to perform a lookup for module relatively from the module that an import statement resides.

A useful feature that \texttt{<importFrom>} provides to Python developer is an ability to specified a module target location. If the module target is prefixed with a dot(.), it will be considered a relative import, where a dot serves as go one level up.


\subsubsection{Import Mechanism}

After the import statement's execution is completed, it results in a set of defined definitions ready to use in the caller's execution context.
To achieve this goal, Python processes import statement in two steps, during the creation of symbol table, which the importing target symbols are created in global execution scope of the import statement caller, then the runtime execution of it is combining searching of the named module and binding result to the importing module's symbol reference.

In the implementation level, these two operations are performed by an abstraction called \texttt{importer}, which consists of two other crucial abstraction objects; finders and loaders.
Finders are responsible for performing module lookup, while loaders use the result from the finder to execute and create a module object which later is saved to the cache.
Lastly, the loader binds name as defined in the import statement to the local execution environment.

Python locates a module by a \texttt{finder}, which is categorized into two types based on the location where the finder should attempt to find the module.
Firstly the \texttt{MetaPathFinder} will perform a lookup at a list of paths passing through the finding method.
The second type of finder, the \texttt{PathEntryFinder}, performs look up only to the path that is assigned during its creation.
Searching operations may cause certain side-effects to the execution environment. In case of importing a module inside the package, it is processed by importing each parent packages from top-level until the module is reached, which means it also executes the initialization instructions of every module target's parent packages.

After the module has been loaded into the execution environment, the importing target symbols are binding to the importer global execution scope; this scope is also called the module scope. There is only a single global scope in python program execution.

\subsubsection{Symbol Privacy}

In Python, every declaration of the symbol is considered a private symbol if it has defined having a prefixed underscore(\_). Such a symbol will be ignored for a wildcard import (from ... import *), and error will be raised when it is implicitly referring to, as the declaration can only be used in its module scope.
