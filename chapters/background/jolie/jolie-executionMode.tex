\subsubsection{Execution modality}

In the previous port definition example at ~\ref{list:example-port}, the runtime execution terminates after service has finished the workflow of \texttt{createPerson} operation. But as a service, we would like it to keep listening to the incoming messages without terminating after the first message has been processed.
Jolie offers a set of primitives of program execution behavior through \texttt{execution} statement which accepts three types of execution behaviors:

\begin{itemize}
    \item \textit{single}, the default behavior, which terminates the program after the current process is finished.
    \item \textit{sequential}, which initiates a new execution whenever there is an operation invocation behavior \texttt{after} the current process is finished.
    \item \textit{concurrent}, which initiates a new execution behavior whenever there is an operation invocation.
\end{itemize}

\begin{figure}[ht]
    \begin{framed}
        \begin{grammar}
            <executionMode>
            ::= `execution' `{' <modality> `}'

            <modality>
            ::= `single'
            \alt `concurrent'
            \alt `sequential'
        \end{grammar}
    \end{framed}
    \caption{Jolie Execution Modality Syntax}
\end{figure}

Thus, it is possible to change the service behavior in figure ~\ref{list:example-port} to accept multiple operations invocation concurrently by adding one single line of code as shown in line 5 of figure ~\ref{list:example-port-concurrent}

\begin{listing}[ht]

    \lstset{language=Jolie,
        style=codeStyle,
        numbers=left,
        firstnumber=1
    }
    \begin{lstlisting}[frame=tlrb, caption= {Jolie concurrent service example}, label={list:example-port-concurrent} ]{typeDef-Jolie}
outputPort LoggerOutputPort { ... }

inputPort personInputPort { ... }

execution {concurrent}

main {
    createPerson(person)(result){
        ...
    }
}
\end{lstlisting}
\end{listing}

\FloatBarrier