\subsubsection{Embedding Services}
\label{sec:embedded}
One of the service composition patterns wildly use is \textit{Orchestration}, which centralizing system logic and behavior in a a single service so called an orchestrator, coordinate and conduct a group of services to achieve a their responsibility together. This can be achieved with ease by embedding mechanism of Jolie, it allows multiple services be executed in a single execution environment, in a hierarchical order.
Moreover, it allows rapid communication of services via local memory communication, which deliver messages internally within the virtual machine.

The embedding mechanism expands a service into two roles; \textit{embedder} and \textit{embedded} service. The embedder conducts services it embedded. This allow the developer to separate a complex Jolie service into a small single responsibility service which can later be reuse in another project. 

The local memory communication channel can be defined by specify communication medium as \textit{local} in a Location field
Jolie offers a handy mechanism to embedded an external service into a local execution. 

Jolie support different technologies to be embedded as a service, as it is not limited to a service defined in Jolie language. Currently Jolie supports two additional services from external technology, which are Java and Javascript. Table ~\ref{table:embedded-technology-path} The defined technology give Jolie program to expect different content in \(\langle Path \rangle\) as defined in table ~\ref{fig:embedded-syntax}

\begin{table}[h]
    \centering
    \begin{tabular}{ |c|l| }
        \hline
        Technology & Expecting content in path                                         \\
        \hline
        Jolie      & A path to a jolie file                                            \\
        Java       & A target to a java class which inherits Jolie's JavaService class \\
        Javascript & A path to a javascript file                                       \\
        \hline
    \end{tabular}
    \caption{Embedding technology and path content}
    \label{table:embedded-technology-path}
\end{table}


\begin{figure}[ht]
    \begin{framed}
        \begin{grammar}
            <embedStmt> ::= `embedded' `{' <embedTechnology> `:' ( <Path> ( `in' <portName> )? )* `}'

            <embedTechnology> ::= 'Java' | 'Javascript' | 'Jolie';

            <path> ::= StringLiteral;

            <portName> ::= <ID>;

            <internalService>
            ::= 'service' <ServiceName> `{' processes `}';
        \end{grammar}
    \end{framed}
    \caption{Jolie Embedding Syntax}
    \label{fig:embedded-syntax}
\end{figure}

The syntax of \(\langle portName \rangle\) defined an output port name to be binding to the embedding service. There are three cases to be considered here, which is detailed in the list below:

\begin{itemize}
    \item If it is left blank, the embedding service will be launch as a standalone service.
    \item If it is pointing to exited outputPort, the embedding service will bind communication channel to embedded service input ports declared \textit{`local'} in location field.
    \item If it is pointing to new outputPort, the embedding service will automatically create a new output port and bind communication channel to embedded service input ports declared \textit{`local'} in location field.
\end{itemize}

The example below in figure ~\ref{list:jolie-embeded} shows an usage of embedded statement in Jolie. Here the communication protocol and the location at line 1 is not defined in code, but it will be bound automatically to the communication port of embedded service before main starts to execute. 

\begin{listing}[ht]

\lstset{language=Jolie,
    style=codeStyle,
    numbers=left,
    firstnumber=1
}
\begin{lstlisting}[frame=tlrb, caption= {Jolie Embedding example}, label={list:jolie-embeded} ]{typeDef-Jolie}
outputPort personService {
    interfaces: personInterface
}

embedded {
    Jolie: "person_service.ol" in personService
}

main {
    ... // person variable initialization omitted
    createPerson@personService(person)(result)
}
\end{lstlisting}
\end{listing}

\FloatBarrier
