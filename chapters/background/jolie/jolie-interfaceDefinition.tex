\subsubsection{Interface Definition}
\label{sec:jolie-interface-def}

The usage of Interface Definition in Jolie is to express a group of operations for message-passing communication. The operations in Jolie can be either a request-response type or a one-way type. The type of operation defines an execution behavior of the service either to it is expected a message back from the server or not. The syntax for interface definition is shown in ~\ref{fig:InterfaceDefinitonSyntax}

\begin{figure}[h]
	\begin{framed}
		\begin{grammar}
			<interfaceDeclaration> ::= `interface' <ID> `{' ( oneWay | requestResponse ) `}'

			<oneWay> ::= (`oneWay' | `OneWay')  `:' <oneWayOp> ( `,' <oneWayOp> )*

			<requestResponse> ::= (`requestResponse' | `RequestResponse') \\ `:' <requestResponseOp> ( `,' <requestResponseOp> )*

			<oneWayOp> ::= <ID> `(' <type> `)' | <ID>

			<requestResponseOp> ::= <ID> `(' <type> `)' `(' <type> `)' <throws>? | <ID>

		\end{grammar}
	\end{framed}
	\caption{Jolie Interface Definition Syntax\protect\footnotemark}
	\label{fig:InterfaceDefinitonSyntax}
\end{figure}

\footnotetext{visit \cite{joliedoc} scope and fault section for an further information on fault handling}

By extending the type we had before, we can declare an interface as following
\begin{listing}[h]

\lstset{language=Jolie,
	style=codeStyle
}
\begin{lstlisting}[frame=tlrb, caption= {Jolie Interface declaration example} ]{typeDef-Jolie}
interface personInterface{
	requestResponse: createPerson(person)(bool)
	oneWay: createPersonAsync(person)
}
\end{lstlisting}
\end{listing}

Here we define an interface of two operations, which required the passing message to be a structure of type `person". When invoking the `createPerson" operation, the calling service will wait and expect a boolean value to be return from the target service to be executed. In contrast, `createPersonAsync' operation will just send the message and continue.

\FloatBarrier
