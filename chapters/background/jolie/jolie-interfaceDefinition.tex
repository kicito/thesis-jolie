\subsubsection{Interface Definition}
\label{sec:jolie-interface-def}

The usage of Interface Definition in Jolie is to define a group of operations for message-passing communication. The operations in Jolie can be either a request-response type or a one-way type. The type of operation defines an execution behavior of the service whether if it is expecting a message back from the server or not. The syntax for interface definition is shown in figure ~\ref{fig:InterfaceDefinitonSyntax}

\begin{figure}[h]
	\begin{framed}
		\begin{grammar}
			<interfaceDeclaration> ::= `interface' <ID> `{' ( oneWay | requestResponse ) `}'

			<oneWay> ::= (`oneWay' | `OneWay')  `:' <oneWayOp> ( `,' <oneWayOp> )*

			<requestResponse> ::= (`requestResponse' | `RequestResponse') \\ `:' <requestResponseOp> ( `,' <requestResponseOp> )*

			<oneWayOp> ::= <ID> `(' <type> `)' | <ID>

			<requestResponseOp> ::= <ID> `(' <type> `)' `(' <type> `)' <throws>? | <ID>

		\end{grammar}
	\end{framed}
	\caption{Grammar for Jolie's interface definition\protect\footnotemark}
	\label{fig:InterfaceDefinitonSyntax}
\end{figure}

\footnotetext{visit \url{https://jolielang.gitbook.io/docs/} at scope and fault section for an further information on fault handling}

By extending the type we had before, we can declare an interface as shown in figure ~\ref{fig:jolie-interface-declaration-example}. We define an interface of two operations, which required the passing message to be a structure of type \texttt{person}. When invoking the \texttt{createPerson} operation, the calling service will wait and expect a boolean value to be returned from the target service to be executed. In contrast, \texttt{createPersonAsync} operation will just send the message and continue.

\begin{listing}[h]

	\lstset{language=Jolie,
		style=codeStyle
	}
	\begin{lstlisting}[frame=tlrb ]{typeDef-Jolie}
	interface personInterface{
		requestResponse: createPerson(person)(bool)
		oneWay: createPersonAsync(person)
	}
	\end{lstlisting}
    \caption{Jolie implementation of interface declaration}
    \label{fig:jolie-interface-declaration-example}
\end{listing}
\FloatBarrier
