\subsubsection{Workflow and Procedure Definition}
\label{sec:jolie-procedure-def}

Jolie execution workflow is defined as a block of execution processes, the initialization block \texttt{init} which is executed once during initialization of the service. While the main execution block \texttt{main} is executed differently based on the execution modifier, as it may executed as a one time execution program, or it may act as a service which waits for an incoming message. The detail on execution mode can be found in section ~\ref{sec:jolie-execution-mode}. The main workflow is essential for a Jolie program as it is an execution target for running a program.

Jolie allows the programmer to split process composition into a procedure that can be invoked by calling its name. Calling the procedure essentially tells the runtime execution to perform the instructions within the caller scope. As an example in figure ~\ref{list:procedure}, calling \texttt{doubleX} in main scope cause \texttt{X} in the procedure to be referred to same \texttt{X} as the \texttt{main} scope. 

\begin{figure}[]
	\begin{framed}
		\begin{grammar}
			<procedureDefinition> ::= `define' <ID> `{' <processes> `}'
		\end{grammar}
	\end{framed}
	\caption{Grammar for Jolie procedure declaration}
\end{figure}


\begin{listing}[]
\lstset{language=Jolie,
	style=codeStyle
}
\begin{lstlisting}[frame=tlrb]{Jolie-procedure}
define doubleX{
	X = X * 2
}

main {
	X = 5
	doubleX
	// now X == 10
}
\end{lstlisting}
\caption{Jolie implementation of procedure calling}
\label{list:procedure}
\end{listing}

\FloatBarrier
