\subsubsection{Procedure Definition}
\label{sec:jolie-procedure-def}

Jolie allows the programmer to split process composition into a procedure that can be invoked by calling its name. Calling the procedure essentially tells the runtime execution to perform the instructions within the caller scope. As an example at ~\ref{list:procedure}, calling \texttt{doubleX} in main scope cause \texttt{X} in the procedure to be referred to same \texttt{X} as the \texttt{main} scope. There are two types of special procedures in Jolie that define target execution processes, which are \texttt{init}, the initialization procedure, and \texttt{main}, the main execution target.

\begin{figure}[h]
	\begin{framed}
		\begin{grammar}
			<procedureDefinition> ::= `define' <ID> `{' <processes> `}'
		\end{grammar}
	\end{framed}
	\caption{Jolie Procedure Definition Syntax}
\end{figure}


\begin{listing}[h]
\lstset{language=Jolie,
	style=codeStyle
}
\begin{lstlisting}[frame=tlrb, caption= {Jolie procedure example}, label={list:procedure}]{procedure-Jolie}
define doubleX{
	X = X * 2
}

main {
	X = 5
	doubleX
	// now X == 10
}
\end{lstlisting}
\end{listing}

The \(<process>\) rule is the composition of tasks to be perform by the service, it will be discuss at \autoref{sec:jolie-behavior}.

\FloatBarrier
