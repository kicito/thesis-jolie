\subsubsection{Type and Value Representation}
\label{sec:jolie-type-def}

The type system in Jolie is based on structural typing\cite{jolie-type-2019}, where the compatibility and equivalence are determined from the actual structure of the comparing target. It is used to define the message's structure both incoming and outgoing. A type can be defined as a nested tree where each node contains a value. Jolie supports common primitive types with an additional two build-in types \textbf{raw} and \textbf{any}. The \textbf{raw} type is an internal type used for passing data, and the programmer cannot create it. The \textbf{any} type indicates that the value can be in any form of the basic types. One of Jolie's data structure's key features is that every variable is a dynamic array.

The type cardinality can be defined to ensure the range of occurrences for a node in the form of \textit{[ min, max ]} where min and max is a positive number that indicates the range. The common quantifier shorthands such as ? and *  are also supported here. If the cardinality is not defined, it will use the default value of [1,1]. As shown in the reference of \texttt{pet} field in figure ~\ref{list:TypeDefinitonUsage}.

The syntax of Jolie type definition grammar is shown in figure ~\ref{fig:TypeDefinitonSyntax}.

A type can be defined in three ways:
\begin{itemize}
	\item \texttt{typeInline} is a structural composition of types, as illustrated in figure ~\ref{list:TypeDefinitonUsage}.
	\item \texttt{typeLink} is a definition aliasing to a previously defined type.
	\item \texttt{typeChoice} is a type that can be represent from multiple types.
\end{itemize}

\begin{figure}
	\begin{framed}
		\begin{grammar}
			<typeDefinition> ::= `type' <ID> `:' <type>

			<type> ::= <typeInline>
			\alt <typeLink>
			\alt <typeChoices>

			<typeChoices> ::= <type> `|' <type>

			<typeLink> ::= <ID>

			<typeInline> ::= <nativeType> <subTypeBlock>?

			<subTypeBlock> ::= `{' `.'? <ID> <cardinality>? `:' <type>? `}'

			<cardinality> ::= `[' int `,' ( int | `*' ) `]'
			\alt `*'
			\alt `?'

			<nativeType> ::= 'void' | 'string' | 'raw' | 'any' | 'int' | 'double' | 'long' | 'bool'
		\end{grammar}
	\end{framed}
	\caption{Grammar for Jolie's type Definition \protect\footnotemark}
	\label{fig:TypeDefinitonSyntax}
\end{figure}

\footnotetext{ID denote an identifier}

\begin{listing}
	\begin{sublisting}{\linewidth}

		\lstset{language=Jolie,
			style=codeStyle
		}
		\begin{lstlisting}[frame=tlrb]{typeDef-Jolie}
type person: int{
	name: void {
		firstname: string
		lastname: string
	}
	isMarried: bool
	pet[0, 2]: string
	age: int
}

person = 1
person.name.firstname = "John"
person.name.lastname = "Doe"
person.isMarried = false
person.pet = "Johnny"
person.pet[1] = "Max"
person.age = 20
\end{lstlisting}
		\caption{ Construction of type and value in Jolie }
		\label{list:type-value}
	\end{sublisting}\\[2ex]
	\begin{sublisting}{\linewidth}
		\lstset{language=XML,
			showspaces=false
		}
		\begin{lstlisting}[frame=tlrb]
<person>
	1
	<isMarried>false</isMarried>
	<name>
		<firstname>John</firstname>
		<lastname>Doe</lastname>
	</name>
	<pet>Johnny</pet>
	<pet>Max</pet>
	<age>20</age>
</person>
\end{lstlisting}
		\caption{ XML representation of \ref{list:type-value} }
		\label{list:type-value-xml}
	\end{sublisting}
	\caption{ Jolie type construction and its representation }
	\label{list:TypeDefinitonUsage}
\end{listing}

\FloatBarrier