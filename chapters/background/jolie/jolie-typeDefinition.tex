\subsubsection{Type and Value Representation}

Type system in Jolie is based on structural type system and is used to define the structure of message of send and receive messages. A type can be defined as a nested tree where each nodes contain value and it's children. Jolie support common primitive types with an additional two build-in types \textbf{raw} and \textbf{any}. The \textbf{raw} type is an internal type that is used for passing data and it cannot be created by the programmer. The \textbf{any} type indicates that the value can be in any form of the basic types. One of the key feature of jolie data storing is that every variable is storing a dynamic array.

As every variable in jolie is a dynamic array, cardinality can be define to ensure the range of occurrences for node in form of \textit{[ min, max ]} where min and max is a positive number indicate the range. The common quantifier shorthands such as ? and * is also support here. If cardinality is not defined, it will use default value of [1,1]. From this specification, an expression referring to a variable without an index is consider as referring to the value at its first index, as shown in the reference of \texttt{pet} field in figure ~\ref{fig:TypeDefinitonUsage}. 

The syntax of Jolie type definition grammar is shown in ~\ref{fig:TypeDefinitonSyntax} where \(<ID>\) is a name defined for it's type structure.

\begin{figure}
	\begin{framed}
		\begin{grammar}
			<typeDefinition> ::= `type' <ID> `:' <type>

			<type> ::= <typeInline>
			\alt <typeLink>
			\alt <typeChoices>

			<typeChoices> ::= <type> `|' <type>

			<typeLink> ::= <ID>

			<typeInline> ::= <nativeType> <subTypeBlock>?

			<subTypeBlock> ::= `{' `.'? <ID> <cardinality>? `:' <type>? `}'

			<cardinality> ::= `[' int `,' ( int | `*' ) `]'
			\alt `*'
			\alt `?'

			<nativeType> ::= 'void' | 'string' | 'raw' | 'any' | 'int' | 'double' | 'long' | 'bool'
		\end{grammar}
	\end{framed}
	\caption{Jolie Grammar\protect\footnotemark}
	\label{fig:TypeDefinitonSyntax}
\end{figure}

\footnotetext{ID denote an Identifier}

\begin{listing}
\begin{sublisting}{\linewidth}

\lstset{language=Jolie,
	style=codeStyle
}
\begin{lstlisting}[frame=tlrb]{typeDef-Jolie}
type person: int{
	name: void {
		firstname: string
		lastname: string
	}
	isMarried: bool
	pet[0, 2]: string
	age: int
}

person = 1
person.name.firstname = "John"
person.name.lastname = "Doe"
person.isMarried = false
person.pet = "Johnny"
person.pet[1] = "Max"
person.age = 20
\end{lstlisting}
\caption{Jolie code for constructing a type and value}
\end{sublisting}\\[2ex]
\begin{sublisting}{\linewidth}
\lstset{language=XML,
showspaces=false
}
\begin{lstlisting}[frame=tlrb]
<person>
	1
	<isMarried>false</isMarried>
	<name>
		<firstname>John</firstname>
		<lastname>Doe</lastname>
	</name>
	<pet>Johnny</pet>
	<pet>Max</pet>
	<age>20</age>
</person>
\end{lstlisting}
\caption{XML representation}
\end{sublisting}
\caption{ Jolie type Example }
\label{fig:TypeDefinitonUsage}
\end{listing}

\FloatBarrier