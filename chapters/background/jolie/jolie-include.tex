\subsection{Include statement}
\label{sec:jolie-include}

Jolie support modular programming via an include statement. Similarly to C programming language include directive, when the parser found an include statement, it substitutes the statement into the content of the target input file. Include statement is used for separate file into a smaller component such as the interface definition file (.iol) or used for adding a functionality of the built in library. The include statement required a string literal indicating the location of the included file. If the given path is an absolute location, Jolie will attempt to find it directly. Otherwise, the searching paths are following.

\begin{itemize}
    \item A relative path resolved from the current working directory, where the jolie interpreter is called
    \item A relative path resolved from Jolie's standard \texttt{include} library directory.
    \item A relative path resolved from paths defined by the passing command line argument.
\end{itemize}

It is worth noting that the searching path is a relative path to the current working directory, not the program itself directory.

Jolie also supports code inclusion from the achieve files.
The including target string can specify as a Java JAR url like \texttt{<scheme>:file:<location>\newline!/{entry}} where \texttt{scheme} can be either a Java archive file (\texttt{jar}) or a Jolie achieve file (\texttt{jap}), \texttt{location} is a path to the archive file, and lastly \texttt{entry} is an internal path to a target inclusion file.

\begin{figure}[h]
    \begin{framed}
        \begin{grammar}
            <includeStmt> ::= `include' StringLiteral
        \end{grammar}
    \end{framed}
    \caption{Jolie Grammar for include statement}
    \label{fig:jolie-definition}
\end{figure}

\subsection{Jolie example program}

Once we are familiar with the basic concept of Jolie, we can demonstrate an example program of Jolie service, a \textit{Sum Service} which take two integers and sum them together. We also going demonstrate a client which attempt to invoke an operation of this service. In this example there are three files involved, a single definition file, and two others are for service declaration.

Firstly, the interface declaration file defined a list of shared definition between the client and Sum Service. It contain a definition of type \textit{intPair} and an interface \textit{SumIface}, which defined an operation that is used for communication.

\begin{listing}[ht]
    \lstset{language=Jolie,
        style=codeStyle
    }
    \begin{lstlisting}[frame=tlrb,caption={Sum service interface}, basicstyle=\footnotesize, label={list:example-iol}]{sumIface.iol}
// sumIface.iol
type intPair: void{
    x : int
    y : int
}
interface SumIface{
    requestResponse: sum(intPair)(int)
}
\end{lstlisting}
\end{listing}

Then we define the sum server shown in figure ~\ref{list:example-sum}, which capable of performing a sum operation when it is called through an input port \textit{IP}. In line 9 of server service, we instruct the service to execute a workflow instruction when receive an invocation. Lastly in our main behavior, we create a non deterministic choice as a handler for the operation defined in \textit{SumIface}.

\begin{listing}[ht]
    \lstset{language=Jolie,
        style=codeStyle,
        numbers=left,
        firstnumber=1
    }
    \begin{lstlisting}[frame=tlrb,basicstyle=\footnotesize, caption= {Sum service}, label={list:example-sum} ]{sumservice.ol}
include 'sumIface.iol' // include interface file

inputPort IP{
    location:'socket://localhost:3000'
    protocol:sodep 
    interfaces:SumIface
}

execution { concurrent }

main{
    [sum(req)(res){
        res = req.x + req.y
    }]
}
\end{lstlisting}
\end{listing}

For our example of the service's client, we firstly include the standard library `console', which exposes us the ability to perform console related tasks eg. print and receive console input. Later at line 4 we declare a communication channel called \texttt{Calculator} which point the communication channel to the channel defined in Sum service. Lastly we declare a behavior of the client, we basically build a message value and send it through invoke a operation \texttt{sum} in the output port.

\begin{listing}[ht]
    \lstset{language=Jolie,
        style=codeStyle,
        numbers=left,
        firstnumber=1
    }
\begin{lstlisting}[frame=tlrb, basicstyle=\footnotesize, caption= {Sum Service client} , label={list:example-client}]{client.ol}
include 'console.iol' // use build-in service 'console'
include 'sumIface.iol' // include an interface file 

outputPort Calculator{
    location:'socket://localhost:3000'
    protocol:sodep 
    interfaces:SumIface
}

define printResult{
    println@Console(result)()
}

main{
    req.x = 2;
    req.y = 3;
    sum@Calculator(req)(result)
    printResult
}
\end{lstlisting}
\end{listing}

\FloatBarrier
