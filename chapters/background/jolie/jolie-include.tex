\subsection{Include statement}
\label{sec:jolie-include}

Jolie supports modular programming via an \say{include} statement. Similarly to the C programming language \say{include} directive, when the parser finds an include statement, it substitutes the statement with the content of the target input file. The include statement is used to separate files into smaller components such as the interface definition file (.iol) or used to add the functionality of the built-in library. The include statement requires a string literal indicating the location of the included file. If the given path is an absolute location, Jolie will attempt to find it directly. Otherwise, the searching paths are the following.

\begin{itemize}
    \item A relative path resolved from the current working directory, where the Jolie interpreter is called
    \item A relative path resolved from Jolie's standard service library directory \texttt{include}.
    \item A relative path resolved from paths defined by the passing command-line argument.
\end{itemize}

It is worth noting that the searching path is a relative path to the current working directory, not the current program's directory.

Jolie also supports code inclusion from Jolie-specific archive files.
The target string can specify as a Java JAR url \textit{<scheme>:file:<location>\newline!/{entry}} where \texttt{<scheme>} can be either a Java archive file (\texttt{jar}) or a Jolie archive file (\texttt{jap}), \texttt{<location>} is a path to the archive file, and lastly \textit{<entry>} is an internal path to a target inclusion file.

\begin{figure}[h]
    \begin{framed}
        \begin{grammar}
            <includeStmt> ::= `include' StringLiteral
        \end{grammar}
    \end{framed}
    \caption{Grammar for Jolie's include statement}
    \label{fig:jolie-include}
\end{figure}

\subsection{Jolie example program}

Now that we are familiar with Jolie's basic concept, we can demonstrate an example program of Jolie service and client, a \texttt{SumService}, which exposes an operation \texttt{sum} that takes two integers and sums them together. There are three files involved in this example, a file containing an interface definition used by the service operation, the sum service program, and the client.

Firstly, the interface declaration file as shown in figure ~\ref{list:example-iol}, defines a list of shared definitions between the client and \texttt{SumService}. It contains a definition of type \texttt{intPair} and an interface \texttt{SumIface}, that encapsulates an operation \texttt{sum} that is used for communication.

\begin{listing}[ht]
    \lstset{language=Jolie,
        style=codeStyle
    }
    \begin{lstlisting}[frame=tlrb]{sumIface.iol}
// sumIface.iol
type intPair: void{
    x : int
    y : int
}
interface SumIface{
    requestResponse: sum(intPair)(int)
}
\end{lstlisting}
\caption{Types and interfaces definitions for sum service, as a content in sumIface.iol }
\label{list:example-iol}
\end{listing}

Then we define the sum service shown in figure ~\ref{list:example-sum}, which is capable of performing a sum operation when it is called through an input port \textit{IP}. At line 9, we specify the service execution mode to keep a service up and running after an operation is invoked. Lastly, in our main procedure, we create an operation handler for the operation \textit{sum} defined in \textit{SumIface}.

\begin{listing}[ht]
    \lstset{language=Jolie,
        style=codeStyle,
        numbers=left,
        firstnumber=1
    }
    \begin{lstlisting}[frame=tlrb]{sumservice.ol}
include 'sumIface.iol' // include interface file

inputPort IP{
    location:'socket://localhost:3000'
    protocol:sodep 
    interfaces:SumIface
}

execution { concurrent }

main{
    [sum(req)(res){
        res = req.x + req.y
    }]
}
\end{lstlisting}
\caption{Jolie implementation of the summation service}
\label{list:example-sum}
\end{listing}

For our example of the service's client, shown in figure ~\ref{list:example-client}, we firstly include the standard library \texttt{console}, which exposes the ability to perform console related tasks, e.g., print and receive console input. Later at line 4, we declare a communication channel name \texttt{Calculator}, which defines the communication channel to the channel in \texttt{SumService}. Lastly, in the client main workflow, we build a message value and send it through invoking an operation \texttt{sum} in the output port.

\begin{listing}[ht]
    \lstset{language=Jolie,
        style=codeStyle,
        numbers=left,
        firstnumber=1
    }
\begin{lstlisting}[frame=tlrb]{client.ol}
include 'console.iol' // use build-in service 'console'
include 'sumIface.iol' // include an interface file 

outputPort Calculator{
    location:'socket://localhost:3000'
    protocol:sodep 
    interfaces:SumIface
}

define printResult{
    println@Console(result)()
}

main{
    req.x = 2;
    req.y = 3;
    sum@Calculator(req)(result);
    printResult
}
\end{lstlisting}
\caption{Jolie implementation of the summation service's client}
\label{list:example-client}
\end{listing}

\FloatBarrier
