\chapter{Jolie Module System}

In this chapter, we present the module system for Jolie.
Similarly to how we presented the Python module system, we will walkthrough the definitions, components, import mechanism, and lastly, the redesign of Jolie's code file layout.

\section{Definitions}

Before delving into the details, it is crucial to have a clear definition of terms used in the Jolie module system. In Jolie:

\begin{itemize}
      \item
            A \textbf{symbol} is a declaration of a file-level abstract node definition. The symbol target is defined along with a module target in the import statement, to specify an imported named definition node.
      \item
            A \textbf{module} is defined as a Jolie program in the file system. A module is a file that contains symbols and the execution target for running the Jolie code.
      \item
            A \textbf{package} is defined as either a directory in the file system or a file. A package role in the Jolie module system is a place to hold the modules and can be part of the module target. A file with \texttt{.jap} extension, the Jolie's JAR-like archive file is considered as a package which it's entry can be referred to as a module.
      \item
            A \texttt{script} is a module which includes a service definition named \texttt{main}. A script can be used as an execution target by the Jolie interpreter. 
\end{itemize}


\section{Import Statement}

An import statement enables the capability of importing a symbol from an external module. The declaration of an import statement consists of two specifiers, the module target and the target symbols, each used in different phases of import statement execution, the module lookup process, and the symbol binding process. Any error that occurs during the execution of an import statement, such as importing a restricted symbol or undefined symbol in the target module, will terminate the interpreter's execution.

The syntax for an import statement is shown in figure ~\ref{fig:jolie-import-stmt-syntax}.

\begin{figure}[ht]
    \begin{framed}
        \begin{grammar}
            <importStmt>
            ::= `from' <importingTargetModule> `import' <importingTargetSymbols>

            <importingTargetModule> ::= ( `.' )* ( <dottedName> | `.' )

            <dottedName>
            ::= <NAME> ( `.' <NAME> )*

            <importingTargetSymbols> ::=  `*' | <importAsNames>

            <importAsNames>
            ::= <importAsName> ( `,' <importAsName> )*

            <importAsName>
            ::= <NAME> ( `as' <NAME> )?

            <NAME> ::= <ID>
        \end{grammar}
    \end{framed}
    \caption{Jolie import statement syntax }
    \label{fig:jolie-import-stmt-syntax}
\end{figure}

The module target specifier used to guide the location where the importing module resides. Jolie import system uses a dot(.) as a path separation to the importing module. Each identifier before last defines the packages and sub-packages where the importing module resides, while the token at the end represents the importing module name. A leading dot designates Jolie import mechanism to perform a relative path lookup.

The second part of the import statement, target symbols, is a list of importing symbol names defined in the target module with an optional qualified name to bind to the local execution environment denoted by using 'as' keyword. For the current implementation of import statement wildcard import doesn't support importing with a qualified name (as).

\section{Symbol and Symbol Table}

Concept of Symbols has been widely used in programming languages. Generally, The term is referred to an unique entities defined and used in the execution environment, the defined entities are such as a variable names, functions names, classes and so on.
The Symbol table is responsible for maintaining a data structure of this information, along with it's description such as the occurrences, declaration scope, and it's type. This data structure is useful for many phases in the complier processes such as the analysis tasks and the optimization.

For the support of module system in Jolie, the Symbol Table is used as a truth table of the declaration symbols in the source code.
It is implemented with hash table and is associate to a single module.
The symbol contains an essential information for performing a binding in later stage this include: symbol name, scope, privacy, and a pointer to abstract syntax node.
Scope indicates location of the declaration node of the certain symbol either Internal or External; internal symbol specify that the symbol definition is declared within the module; on contrary, external symbol, created from an import statement, specify that the definition is declared from the external source.
Privacy is used for controlling the access of the symbol from the external module. It can be defined as \texttt{public}, which allow access from external module, on contrary, \texttt{private} prohibit the access from external. A symbol will be set as public by default; it can be defined as private by adding `private' keyword.

As a result of symbol table introduction, we have the definitions abstract syntaxes as defined in ~\ref{sec:jolie-def} and additionally, the Service node, implemented as a symbols. The Service node is a new abstract syntax of Jolie for declaring a service, which we will get back to in later section.

The construction of symbol target consists of two stages; first stage, each node of definitions in AST get visited and assign to the symbol table, at the completion of this stage, every symbols that is defined within the module get assign its binding declaration abstraction node; on the second stage, executed after every related modules symbol table are finish their first stage creation, the entry of external scope record get assigned their corresponding abstract syntax node by a simple name lookup to the referring module's symbol table.

\begin{figure}[ht]
    \begin{subfigure}[b]{\textwidth}
        \lstset{language=Jolie,
            style=codeStyle,
            numbers=left,
            firstnumber=1
        }
        \begin{lstlisting}[frame=tlrb,
            basicstyle=\footnotesize]{symboltable.ol}
from C import *
from .B import b_type as b_imported
type A {
    a_type: int
}
private type from_b: b_imported
\end{lstlisting}
    \end{subfigure}
    \begin{subfigure}[b]{\textwidth}
        \begin{tabular}{ |c|l|l|l| }
            \hline
            Symbol name & scope        & privacy & binding abstract syntax \\
            \hline
            C*          & external(C)  & private & NONE                    \\
            b_imported  & external(.B) & private & NONE                    \\
            A           & local        & public  & $<$ type: A $>$         \\
            from_b      & local        & private & $<$ type: from_b $>$    \\
            \hline
        \end{tabular}
    \end{subfigure}
    \caption{Jolie code and its Symbol table}
    \label{fig:jolie-ex-symbol-table}
\end{figure}

From figure ~\ref{fig:jolie-ex-symbol-table} which demonstrates the construction of Symbol table.
There are three entries gets created at first step of symbol table creation. The detail of each symbol is following:

\begin{itemize}
    \item \texttt{C*}, originated from an import statement, is instantiated as an \texttt{external(C*)} which determine that the definition node is referring to an external source module named B. In this first step, the symbol name get assigned as a unique name which to be removed later.
    \item \texttt{b_import}, originated from an import statement, is instantiated as an \texttt{external(.B)} which determine that the definition node is referring to an external module named B at relative location respect to execution node.
    \item \texttt{A}, as being defined within local execution, is a local scope symbol with public, the default privacy.
    \item \texttt{from_b}, however, has a keyword private before the actual declaration, thus, this symbol cannot be import by others.
\end{itemize}

After this step, the partially created symbol table of the module is created. This partially filled symbol table has an information of binding abstract syntax which is defined within its module. While the external scope symbol are left without pointer to a binding abstract syntax. In order to complete this field, we can simply perform a lookup to symbol table of the module target that the declaration is originated.

Consider figure ~\ref{fig:jolie-ex-symbol-table} which defined a symbol of module \texttt{C} and the completed symbol table from previous example. Entry \texttt{C*} get replaced by exposed symbol in its module.

\begin{figure}[ht]
    \begin{subfigure}[b]{\textwidth}
        \begin{tabular}{ |c|l|l|l| }
            \hline
            Symbol name & scope & privacy & binding abstract syntax   \\
            \hline
            C1          & local & public  & $<$ type: C1 $>$          \\
            C2          & local & private & $<$ type: C2 $>$          \\
            CIface      & local & public  & $<$ interface: CIface $>$ \\
            \hline
        \end{tabular}
    \end{subfigure}
    \begin{subfigure}[b]{\textwidth}
        \begin{tabular}{ |c|l|l|l| }
            \hline
            Symbol name & scope        & privacy & binding abstract syntax        \\
            \hline
            C1          & external(C)  & private & $<$ type: C1 $>$ in C          \\
            CIface      & external(C)  & private & $<$ interface: CIface $>$ in C \\
            b_imported  & external(.B) & private & $<$ type: b_type $>$ in B      \\
            A           & local        & public  & $<$ type: A $>$                \\
            from_b      & local        & private & $<$ type: from_b $>$           \\
            \hline
        \end{tabular}
    \end{subfigure}
    \caption{Jolie Symbol table after completion}
    \label{fig:jolie-ex-symbol-table}
\end{figure}

\subsubsection*{Module Record}

From here, we can defined a new abstraction object in Jolie interpreter ecosystem. Considering that the symbol table is a representation of symbol usage in a single module, and a module is parsed into a single AST. Thus, for the simplicity of both implementation an the explanation for later in this report. The Module record is an object consisted of an AST and symbol table of a module it's referred to.


\section{ Name Resolution }

The introduction of Symbol Table allows Jolie interpreter parses the definition referencing in abstract syntax by using lazy evaluation strategy. This benefits Jolie developer by allowing symbol referencing without worry about definition order. To illustrate the problem Jolie had before, consider the following source code:
\begin{listing}[ht]
    \lstset{language=Jolie,
        style=codeStyle
    }
    \begin{lstlisting}[frame=tlrb]{definitionOrder.ol}
define bar{
    foo
}

define foo {
    bar
}
\end{lstlisting}
\end{listing}

The execution is failed to parse since the parser cannot find a procedure definition \texttt{foo} when attempt to parse token foo inside procedure \texttt{bar}.
By introducing name resolution to the current workflow, it is possible to write and use definition without a concern on the order. During parsing phase, Jolie interpreter assigns an empty named symbol object which will be resolved to an expected abstract syntax by performing name resolution.

l
\section{ Module Crawling }

In order to fulfill the value in binding abstract syntax of \texttt{b_import} in out previous example. The crawling procedure, which constructs a set of required Module Records for the execution of a program, is fulfilled by a Module Crawler. It is an object introduced in Jolie module system, which consists of module finders and program parser.
The crawling process take an initial module record and scan for dependent module symbol, particularly one with external scope, attempt to find and parse the depending module target until every module record is available for the second step of filling module table entry. The pseudo code for it's procedure is defined in procedure \texttt{crawl} in figure ~\ref{algo:crawl}.

\begin{algorithm}[h]
    \caption{Crawl module}
    \label{algo:crawl}
    \begin{algorithmic}[1]

        \Require{A crawling module record $root$}
        \Ensure {A hash table of required module records $result$}

        \Procedure {Crawl}{$root$}

        \State $Q\gets \Call{FetchDependency}{root}$ \Comment{module to crawl}
        \State \Call{$result.put$}{$root$} \Comment{add root to result}

        \While{$Q$ is not empty}
        \State $module\gets \Call{Q.poll}{}$

        \If{$module$ in $result$}
        \textbf{continue} \Comment{record is already parsed}
        \EndIf

        \State $record\gets$ \Call{parse}{$module$} \Comment{parse source into a module record}

        \State $deps\gets \Call{FetchDependency}{record}$
        \ForAll {$d$ in $deps$}
        \If{$d$ not in $Q$}
        \State \Call{$Q.add$}{$d$}
        \EndIf
        \EndFor
        \State \Call{$result.put$}{$record$} \Comment{add record to result}

        \EndWhile
        \EndProcedure



        \Function {FetchDependency}{$mr$}
        \State $deps\gets$ new list
        \ForAll{$s$ in $mr$'s external scope symbols}
        \State $moduleTarget\gets$ module target of $s$
        \State $source\gets \Call{find}{moduleTarget}$ \Comment{find operation by finder}

        \State \Call{$deps.insert$}{$source$}
        \EndFor

        \State \textbf{return} $deps$ \Comment{list of dependency module for $mr$}
        \EndFunction

    \end{algorithmic}
\end{algorithm}
\FloatBarrier


Module finding is performed by sub-component of the module crawler called finder. The current implementation of Jolie module system has two type of finder; an Absolute Finder, which perform a lookup at a set of defined path locations; and the Relative Finder, which perform a lookup from a path relatively to the import statement caller. The Finder performs module finding task and make sure that the import target is exists in file system, if the finding failed, module not found error is raised and the interpreter should stop it's execution.

\subsection{ Absolute Path Resolution }

For absolute import, which perform a search for the module at a defined path locations, is used for importing a library which is not reside in the package specific modules. The use case for such import is the build-in modules.

The absolute path finder performs lookup in two different locations; working directory of the execution process is the first place to perform lookup, where finder directly tries to resolve the target module. then continue to the directory \texttt{lib} to look for \texttt{first.jap} where \texttt{first} is a first token defined in module target. Lastly, the finder will perform module lookup to the extensible list of directories defined in special path called \texttt{packLib}. It contains a path to a build-in libraries and can extend through command-line arguments. The procedure for the absolute path resolution is shown in figure ~\ref{algo:absolutePath}:

\begin{algorithm}[h]
    \caption{AbsolutePathResolution Procedure}
    \label{algo:absolutePath}
    \begin{algorithmic}[1]
        \Require{A module target $target$}
        \Require{A working directory of execution process $wdir$}
        \Require{A list of package directories $libs$}

        \Ensure {A location to the existed module in file system}

        \Procedure {find}{}
        \State $P\gets$ \Call{$target.split$}{`.'} \Comment{package/module tokens}
        \State $first\gets P[0]$ \Comment{first element of $P$}
        \State $rest\gets P[1...]$ \Comment{elements after first of $P$}

        \State $module \gets$\Call{resolve}{$wdir$, $P$}
        \If{$module$ existed}
        \State \textbf{return} $module$ \Comment{target module located in $wdir$}
        \EndIf

        \State $japPack \gets$\Call{resolve}{$wdir$, `lib', $first$}
        \If{$japPack$ existed and has jap extension}
        \If{$japPack$ has $rest$ as a entry}
        \State \textbf{return} $japPack$
        \EndIf
        \EndIf

        \ForAll{$lib$ in $libs$}
        \State $module \gets$\Call{resolve}{$lib$, $P$}
        \If{$module$ existed}
        \State \textbf{return} $module$ \Comment{target module located in $lib$}
        \EndIf
        \EndFor

        \State \textbf{return} None \Comment{lookup failed}

        \EndProcedure

    \end{algorithmic}
\end{algorithm}

Consider figure ~\ref{fig:jolie-absolute-import}, which demonstrates a working directory tree representation and a working code snippet in \texttt{main.ol}, shows the relation between module targets to the actual working directory structure. The import statement defined in first line get resolve into a ./packages/somePackage.ol, while second line demonstrates an example of importing a jap archive file. Lastly the last line express an example of importing a built-in module \texttt{console}.

\begin{figure}[]
    \begin{forest}
        for tree={
        font=\ttfamily,
        grow'=0,
        child anchor=west,
        parent anchor=south,
        anchor=west,
        calign=first,
        edge path={
                \noexpand\path [draw, \forestoption{edge}]
                (!u.south west) +(7.5pt,0) |- node[fill,inner sep=1.25pt] {} (.child anchor)\forestoption{edge label};
            },
        before typesetting nodes={
                if n=1
                    {insert before={[,phantom]}}
                    {}
            },
        fit=band,
        before computing xy={l=15pt},
        }
        [.
            [lib
                    [someLib.jap
                            [module.ol]
                    ]
            ]
            [packages
                    [somePackage.ol]
            ]
        main.ol
        ]
    \end{forest}

    \lstset{language=Jolie,
        style=codeStyle,
        numbers=left,
        firstnumber=1
    }
    \begin{lstlisting}[frame=tlrb, caption={Demonstration of absolute import usage in jolie}, label={fig:jolie-absolute-import}]{import-statement-absolute.ol}
from someLib.module import componentA
from packages.somePackage import componentB
from console import consoleService
\end{lstlisting}
\end{figure}

\FloatBarrier


\section{Jolie Module System Implementation}

In this section, we are looking into workflow of Jolie interpreter for support of the module system. After discussions on necessary components for an import mechanism in Jolie, we are ready to revisit the flow we had mention before in Section ~\ref{sec:jolie-implementation}; bold typesetting is used for indicate the additional flow.

\begin{enumerate}
    \item The command-line configuration get parsed, this include locate the executing target file.
    \item The source code get parsed in parser, where the \textbf{module record of execution target is created}.
    \item \textbf{The dependent modules gets crawled by Module Crawler, given the initial module record.}
    \item \textbf{The symbol tables are revisited and external symbol entries gets fully filled.}
    \item \textbf{The missing program's reference get solve from name resolution.}
    \item The program's component nodes in AST get verified it's semantic.
    \item The interpretation tree, or the runtime executable object, is generated from AST \textbf{of the main service definition}.
    \item The CommCore is initialized from the communication ports configuration declared in AST node.
    \item The embedding services is initialized and communication related variable is set.
    \item The runtime execute init/main procedure definition process.
\end{enumerate}

Apart from the additional phases for symbol tables and name resolution, the resolving alias type which was perform during  verification phase is no longer needed; since the task is already handled during name resolution process.
Another notable change in the workflow is the target of OOIT creation where we used to pass the whole AST of a Jolie file.
This is no longer be the case, since our definition of a Jolie executable file has changed from being a representation of a service into a module.
Hence, we introduce a \textit{service definition} as an input of the interpretation tree creation process.

\section{Service Node}

As the adjustment of Jolie execution file definition has changed, the file structure layout has to be modified as well.  At top level definitions, the abstract syntax of definition nodes such as types and interfaces are remained. But the service related syntax nodes such as behavioral and instruction on deployment has to be moved into a proper scope which represent a definition of a service. This lead to an introduction of a new syntax node in Jolie module system; such node is called a \textit{Service Node}.

In the implementation level, Service Node is responsible for storing deployment instructions and behaviors of a service. It may optionally accepted a parameter, which can be passed to during embedding process.
With this change, we can use the import mechanism to import and embed a service easily, by specify a service as an identifier, rather than using string defining a path as before.
The definition of service node also covers declaration of the external technology service, in particular, the one that we used in existed embedding mechanism ~\ref{sec:embedded}.
In this new implementation of foreign technology service node, it can be view as a special type of jolie service which can interact with foreign technology.
The grammar for Jolie module system is defined in figure ~\ref{fig:jolie-servicenode-grammar}

\begin{figure}[h]
    \begin{framed}
        \begin{grammar}
            <jolie> ::= <importStmt>* <definitions>*

            <definitions> ::= `private'? <definition>

            <definition> ::=  <typeDefinition>
            \dots
            \alt <serviceNode>

            <serviceNode> ::= `service' <technology>? <ID> <param>? \\ `{' <deplInstruction>* `main' `{' <behaviors> `}'`}'

            <technology> ::= <java> | <javascript> | <jolie>

            <jolie> ::= `Jolie'

            <java> ::= `Java' `(' <StringLiteral> `)'

            <javascript> ::= `Javascript' `(' <StringLiteral> `)'

            <param> ::= `(' <ID> : <paramType> `)'

            <paramType> ::= <ID>
        \end{grammar}
    \end{framed}
    \caption{Jolie Service Node Grammar}
    \label{fig:jolie-servicenode-grammar}
\end{figure}

\begin{listing}[h]

    \lstset{language=Jolie,
        style=codeStyle,
        numbers=left,
        firstnumber=1
    }
    \begin{lstlisting}[frame=tlrb, caption= {Jolie Service Node Example}, label={list:jolie-servicenode} ]{servicenode-jolie}
interface prefixerIface{
    requestResponse: prefix(string)(string)
}

service prefixService ( prefix: string )  {
    
    inputPort IP {
        interfaces : prefixerIface
        protocol: sodep
        location : "socket://localhost:3000"
    }

    execution{ concurrent }

    main {
        [prefix(req)(res){
            res = prefix + req
        }]
    }
}
    \end{lstlisting}
\end{listing}

Listing ~\ref{list:jolie-servicenode} illustrates a code snippet for declaration of service node, \textit{prefixService}. The service accept a string as a parameter 

It is in clear view that the definitions that direct the behavior of a service is encapsulate in a Service Node called \textit{mulService}, This service accepted a value of integer, which determine a second factor of multiplication operation. It is also exposes the communication channel through an interface \textit{mulIface}, which is declared at file level scope and can be imported by an external module.

\FloatBarrier

\subsubsection*{Jolie Script}

Running a Jolie interpreter instance involve a translation of a service abstraction into an OOIT as one of the execution process.
As we have changed the definition of a service, in particular, a Jolie executable file is no longer represent a service, rather becoming a module.   
We also introduce an encapsulation of the service abstraction into a service node, which multiple can be defined within a single module.
For these reasons, making it ambiguous for an interpreter to decided which service node to be used as an translation target, it is essential to have a convention of target service node when a module is given as a execution target.
Hereby, we introduce the definition of an executable Jolie module so called \textit{Jolie script}.

Jolie script, a main candidate for running Jolie program, is a module that included a service node named \texttt{main}. Considering the listing ~\ref{list:jolie-procedure-script} which is a rewritten version of the example we had before in ~\ref{list:procedure}, the only difference here is the behavior of the service is enclosed in a special service node named main. Which is a execution target when the script get pass to the command-line.

It is worth mentioning that running a non Jolie script modules, similar to running a Jolie program without a main procedure, is prohibited and program will terminate during semantic verifying step.

\begin{listing}[h]
    \lstset{language=Jolie,
        style=codeStyle
    }
\begin{lstlisting}[frame=tlrb, caption= {A Jolie script version of ~\ref{list:procedure}}, label={list:jolie-procedure-script}]{procedure-script-Jolie}
define doubleX{
    X = X * 2
}

service main {
    X = 5
    doubleX
    // now X == 10
}
\end{lstlisting}
\end{listing}

\subsection{Embedding Service Node}

As the definition of service has changed, we also needed to take a careful look at the embedding mechanism for supporting a service node. As per a distinct syntax between current Jolie implementation where it defined the embedding node through a string literal to service path in file system. We separated a new statement, so-called \textit{embed}, for the declaration of embedding a service node. The syntax for declaring an embed statement is shown in figure ~\ref{fig:jolie-servicenode-embed-grammar}.

            % embed someservice2(ConstLiteral) in new OP2

\begin{figure}[h]
    \begin{framed}
        \begin{grammar}
            <deplInstruction> ::= ... 
            \alt <embedStmt>

            <embedStmt> ::= `embed' <serviceName> <args>? <bindingPort>?  

            <args> ::= `(' <expr> `)'

            <serviceName> ::= <ID>

            <bindingPort> ::= `in' `new'? <portID>

            <portID> ::= <ID>

        \end{grammar}
    \end{framed}
    \caption{Embedding service node grammar}
    \label{fig:jolie-servicenode-embed-grammar}
\end{figure}

The declaration of the new embedding statement for a service node can be declared in three ways, depending on the need of binding a port between the embedder and the embedded service. The jolie developer can explicitly define a behavior of the embedding through a straightforward syntax similar to the original one. 

In order to understand the implementation of the embedding service node. Firstly, the embedding a service required passing a value that is a type declared in the embedder's parameter. if the value is omitted, the interpreter will pass a value of \textit{undefined} to a service node. If the value has a valid structure as defined constrain in the target embedding node. Following cases are consider:

\begin{itemize}
    \item if the binding port is not defined, the embedding service is initiate as a standalone service that doesn't has any binding communication channel between embedded and embedding service.
    \item if the binding port is defined using an existed port, the embedding service is initiated with bounded communication channel between it's incoming port declared using local memory and the binding port in the embedder.
    \item if the binding port is defined using \textit{new} keyword, the embedding service is initiated with bounded communication channel between it's incoming port declared using local memory and the newly created outgoing port in the embedder.
\end{itemize}

The evolution of embedding mechanism statement, by using a symbol referring to a service instead a string path, also helps Jolie's interpreter eliminates an error that might occur during resolving a target service path which was checking during runtime of the execution. As the abstraction node is resolved during name resolution step, this error is caught preemptively during before runtime execution. Moreover, it also helps generalizing path resolution algorithm for Jolie internal statements into one place, namely through the import statement.

\subsubsection{Parameterize Service}

A service node of Jolie is parameterize. 

\subsection{Backward compatibility}

Since Jolie module system is relatively new and cause a breaking change on the previous version, we decided to keep the backward compatible with the old syntax. If the interpreter detected old syntax or jolie file, the AST will be transform by moving all behavior and deployment instruction nodes into the newly created main Service Node while definition nodes will remain in top level definitions. The method of detecting old syntax is relatively easy, one of trivial method is to detect whether the AST contain an Procedure Definition named \texttt{main}.

\begin{algorithm}[h]
    \caption{TransformJolieCodeToModule}
    \label{algo:transfrom}
    \begin{algorithmic}[1]
        \Require{An abstract syntax tree of Jolie program $program$}
        \Ensure {An abstract syntax tree of Jolie Module $result$}

        \Procedure {transform}{}
        \State $moduleProgram\gets$ new list\Comment{new module's program node}
        \State $serviceProgram\gets$ new list\Comment{new new Service program node}

        \ForAll{$node$ in $program$}
        \If{$node$ is $Symbol$}
        \If{$node$ is $Procedure$ and $node$ named \texttt{`main"} or \texttt{`init"} }
        \State \Call{$serviceProgram.add$}{$node$}
        \Else
        \State \Call{$moduleProgram.add$}{$node$}
        \EndIf
        \Else{
            \If{$node$ is $ImportStatement$}
            \State \Call{$moduleProgram.add$}{$node$}
            \Else
            \State \Call{$serviceProgram.add$}{$node$}
            \EndIf
        }
        \EndIf
        \EndFor
        \State $serviceNode \gets$ new service node from $serviceProgram$
        \State \Call{$moduleProgram.add$}{$serviceNode$}
        \State $result \gets$ $moduleProgram$
        \EndProcedure

    \end{algorithmic}
\end{algorithm}
\subsection{Code example}

In this section, we introduce the reader to a simple example program of Jolie module system to a sophisticated design pattern \textit{Service Injection Pattern}, which is a dependency-injection like design pattern made possible from the implementation of the module system.

The following code snippet is a hello-world example of the Jolie module system.

\begin{listing}[H]
    \lstset{language=Jolie,
        style=codeStyle,
        numbers=left,
        firstnumber=1
    }
    \begin{lstlisting}[frame=tlrb,
basicstyle=\footnotesize]{service-hello.ol}
from console import ConsoleService, ConsoleInterface

service main(){

    outputPort Console {
        interfaces: ConsoleInterface
    }

    embed ConsoleService() in Console

    main{
        print@Console("Hello")()
    }
}
\end{lstlisting}

\end{listing}

At line 1, we declared an import statement to include the ConsoleService along with the interface exposed by this service. The ConsoleService is a build-in service that allows a Jolie program to communicate to the execution process's system console. Afterward, we declare a service node called \textit{main}, which consists of an output port with an assignment of operations in the ConsoleInterface binding to the embed service of the ConsoleService, and the main procedure of the service. 
In our main procedure, we send an operation \textit{print} through the output port Console. Executing this program resulted in Hello is printed on the process console.

\subsubsection{Service Injection Pattern}

Our last example describes a way to compose a simple service in the Jolie module system. Here we can extend the previous example to an advanced scenario. Given a situation where we want to have our workflow on the operation \textit{print}, the developer can create a new service, \textit{PrinterService}, which exposes the operation \textit{print} and prefixing an incoming message before forwarding it to the system console.

\begin{listing}[H]
    \lstset{language=Jolie,
        style=codeStyle,
        numbers=left,
        firstnumber=1
    }
    \begin{lstlisting}[frame=tlrb,
basicstyle=\footnotesize]{printer.ol}

from console import ConsoleService, ConsoleInterface

interface PrinterInterface {
RequestResponse:
    print( undefined )( void )
}

service PrinterService() {

    execution { concurrent }
    
    inputPort IP {
        location: "local"
        interfaces: PrinterInterface
    }

    embed ConsoleService in new _Console

    main{
        [print(req)(res){
            res = "Printer receive: " + req
            print@_Console(res)()
        }]
	// omitted code
    }
}

\end{lstlisting}

\end{listing}

After the service has defined, the developer only has to import and embed \textit{PrinterService} instead of \textit{ConsoleService} in our last example. The client of this service will now point the operation target to the \textit{PrinterService}.

\begin{listing}[H]
    \lstset{language=Jolie,
        style=codeStyle,
        numbers=left,
        firstnumber=1
    }
    \begin{lstlisting}[frame=tlrb,
basicstyle=\footnotesize]{service-injection-hello.ol}
from console import ConsoleInterface
from .printer import PrinterService

service main(){

    outputPort Console {
        interfaces: ConsoleInterface
    }

    embed PrinterService() in Console

    main{
        print@Console("Hello")()
    }
}
\end{lstlisting}
\end{listing}

Since the \textit{PrinterService} operation \textit{print} is defined as a subset of operations in \textit{ConsoleService}. Thus, these two interfaces are compatible to each other.

The module system extends the flexibility of the way we implement a service in Jolie and allows the service developer to create a new system in a modular approach. In the next section, we will look at the practical Jolie module system implementation of the Circuit Breaker, which is one of the prominent  Microservice pattern. 

\subsubsection{Microservice Pattern: Circuit Breaker}
In this section, we discuss an implementation of the circuit breaker pattern in Jolie and emphasize Jolie's capability as a language for microservices. After the discussion on implementation source code, we also explore the integration of this application with the container technology, particularly Docker container.

The circuit breaker is a well-known pattern in the Microservice Architecture. It enables an application to proceed on a request properly on the situation of an arbitrary dependent service that has failed to progress. Given a situation where the network of service communicates with each other to build up the client's response, and one of the services is making a longer than usual processing time. This processing time can be a result of the network latency, or there might be a failure during processing, which cannot foresee by other services. The circuit breaker pattern is a pattern addressing a problem when a dependent service fails to process the request either from the longer than usual processing time or an unexpected error thrown from the invocation. The pattern increases the fault-tolerant of the whole system and helps the service maintainer investigate the issues precisely.

A circuit breaker implementation in the Microservice ecosystem is a proxy between a client to a service that might have failed to deliver the response. The circuit breaker monitors the request and counts the number of failures that happen through calls made to the destination service. If the number of failures reaches the threshold, it discards and returns a meaningful response for the client without attempting to make a call to the destination service. Later after a certain amount of time, on the invocation to the service, the circuit breaker slowly forwards the request again.

There are three states in Circuit breaker which determine the state of calls made to the destination service. Firstly, the \textit{Closed} state is a healthy state of the circuit breaker; at this state, every request for an operation will be passed to the service. Each failure occurred from the call, either internally from the service or the request timeout, the circuit breaker increases the counter. When the number of failures reaches the threshold number, the circuit breaker trips by changes its state to \textit{Open} state and start a reset timeout. At this state, any request making to the destination operation is skipped by the circuit breaker as it is presumed that the destination service is down. The developer of the circuit breaker can design the failback procedure, for example, return a cached response to the client or straightforwardly return an error response. After the reset timeout tick, the circuit breaker changes its state to \textit{Half-Open}, where any call will be passed to the service destination. If the service responds without any error, the circuit breaker sets its state to \textit{Close} or in a healthy state. Otherwise, the circuit breaker falls back to status \textit{Open} again. The states for a circuit status illustrate in the figure ~\ref{list:circuit-breaker}.

\begin{figure}[ht]
    \includegraphics[width=10cm]{CircuitBreaker}
    \centering
    \caption{A state diagram for Circuit Breaker}
    \label{list:circuit-breaker}
\end{figure}

Montesi and Weber have studied the implementation of the circuit breaker pattern for Jolie application\cite{10.1145/3167132.3167427}, which extends a decoration pattern and sketch an implementation of the circuit breaker pattern in Jolie. In our version, we exploit the capability of importing service to extend a single service with a circuit breaker extension service.

Start with a circuit breaker service, which implemented as an extension service to be imported and embedded by others. We firstly define definitions for the circuit breaker. As a requirement above, we define two communication ports to manage the functionality of the service, one for managing the communication between the circuit breaker to the destination service for handling the internal operation such as timeout for resetting the \textit{Opened} state and request timeout handler. Furthermore, we also define a type for configuring the service. As a proxy service, we require additional input from the embedder service, the exposing input location, and the target location. The source code for what we have mentioned so far is show at ~\ref{list:circuit-breaker-skel-application}

The client requests are handled through an HTTP protocol with a special configuration \textit{default}, which will be invoked as a default operation when the undefined operation is passed to the service. It is worth mentioning that we give the authority to importer service to configure \textit{location} of our \textit{HTTPInput} and \textit{DestService} field by assigning the service parameter.


\begin{figure}[ht]
    \includegraphics[width=10cm]{CircuitBreakerDiagram}
    \centering
    \caption{A service diagram for Circuit Breaker application}
    \label{list:circuit-breaker-skel-application}
\end{figure}

\begin{listing}[H]
    \lstset{language=Jolie,
        style=codeStyle,
        numbers=left,
        firstnumber=1
    }
    \begin{lstlisting}[frame=tlrb,
basicstyle=\footnotesize]{circuitbreaker-definitions.ol}
interface HTTPInterface{
    requestResponse:
        default(undefined)(undefined)
}

interface CircuitBreakerServiceCallbackInterface{
    requestResponse: callback(undefined)(undefined) throws UnexpectedError(string)
}

type CircuitBreakerServiceParam : void {
    inputLocation: string
    outputLocation: string
}

private interface CircuitBreakerInternalInterface{
    oneWay: reset(undefined)
}

service CircuitBreakerService (p : CircuitBreakerServiceParam) { 
    inputPort HTTPInput {
        protocol: "http" {
            default = "default"
        }
        location: p.inputLocation
        interfaces: HTTPInterface
    }

    outputPort DestService {
        location: p.outputLocation
        protocol: "sodep"
        interfaces: CircuitBreakerServiceCallbackInterface
    }

    inputPort SelfInput{
        location: "local"
        interfaces: CircuitBreakerInternalInterface
    }

    ...

}

\end{lstlisting}
\end{listing}

Next, we define procedures and workflow for each operation exposed by the circuit breaker. The procedures included in this service encapsulate the actions made for state transformation. We use a global state, a shared state lived in every session initiated by the service, to store the counter of error number and the flag of the current status of the circuit breaker.

    \lstset{language=Jolie,
        style=codeStyle,
        breaklines=true,
        columns=fullflexible,
        numbers=left,
        breaklines=true,
        firstnumber=1
    }
    \begin{lstlisting}[frame=tlrb,
basicstyle=\footnotesize]{circuitbreaker-port.ol}
constants {
    STATUS_CLOSED = 1,
    STATUS_OPEN = 2,
    STATUS_HALFOPEN = 3,
    ERROR_THRESHOLD = 3,
    TIMEOUT_REQUEST = 1000
}

service CircuitBreakerService (p : CircuitBreakerServiceParam) {
    ...
    define closeCircuit {
        global.errorCount = 0
        global.status = STATUS_CLOSED
    }

    define resetCircuit {
        global.status = STATUS_HALFOPEN
    }

    define trip {
        global.status = STATUS_OPEN
        scheduleTimeout@Time( TIMEOUT_REQUEST{.operation="reset"} )( )
    }

    define handleError {
        if (global.status == STATUS_HALFOPEN){
            trip
        } else if (global.errorCount > ERROR_THRESHOLD){
            trip
        }
    }

    execution { concurrent }

    init {
        closeCircuit
    }

    main {        
        [ reset() ]{
            resetCircuit
        }
        [ default( request )( response ) {
            if ( global.status == STATUS_OPEN ){
                response = "CircuitOpen"
            } else {
                install( UnexpectedError =>
                    global.errorCount++
                    handleError
                    response = "ServiceError"
                )
                callback@DestService(request)(serviceRes)

                if (global.status == STATUS_HALFOPEN){
                    closeCircuit
                }

                response << serviceRes
            }
        }]
    }
}
\end{lstlisting}

The \textit{trip} procedure defines the instruction set to turn on the circuit, which is determined by procedure \textit{handleError}. After modifying the state, it proceeds with scheduling the timeout request for state recovery, defined in \textit{resetCircuit}, by invoking the operation from build-in service \textit{time} \footnote{Time service specification https://jolielang.gitbook.io/docs/language-tools-and-standard-library/standard-library-api/time}. 

The circuit breaker's initialization begins with a call to procedure \textit{closeCircuit}, which sets the state of the circuit to \textit{Closed} state. Later in the main execution, the service exposes and wait for two operations \textit{reset} and \textit{default}. The \textit{reset} operation is responsible for turn an Opened state of the circuit breaker to \textit{Half-Opened} by calling \textit{resetCircuit} procedure defined above. The operation for handling client requests, \textit{default}, as mention above, is invoked whenever the client is passing a message to the circuit breaker and forward the request to the destination service. The workflow of operation can be described as the following:
If the state of the circuit breaker, return a response without passing the request to destination service.
Otherwise, forward the request to destination service, if there is an error occur, call the error handling procedure and return a response to the client.

After we have defined the CircuitBreaker service, which can be imported and embed by any services, we look into an example service \textit{AddService}, which integrates the CircuitBreaker extension to handle the possible error that might occur internally. 

\begin{listing}[ht]
    \lstset{language=Jolie,
        style=codeStyle,
        numbers=left,
        firstnumber=1
    }
    \begin{lstlisting}[frame=tlrb,
basicstyle=\footnotesize]{circuitbreaker-client.ol}
from .circuitbreaker import CircuitBreakerService, CircuitBreakerServiceCallbackInterface

type input: void{
    x : int
    y : int
}

interface calculatorIface {
    requestResponse: 
        add(input)(int)
}

service main {

    inputPort SelfIn {
        location: "local://self"
        interfaces: calculatorIface
    }

    outputPort SelfOp {
        location: "local://self"
        interfaces: calculatorIface
    }

    inputPort fromCircuitBreaker{
        location: "socket://localhost:3001"
        protocol: "http"
        interfaces: CircuitBreakerServiceCallbackInterface
    }

    embed CircuitBreakerService( { 
        inputLocation = "socket://localhost:3000"
        outputLocation = "socket://localhost:3001" 
    } )

    execution { concurrent }

    main{
        [add(req)(res){
		...
        }]
        [callback(req)(res){
                    if (req.operation == "add"){
                        add@SelfOp({x = int(req.data.x) y = int(req.data.y)})(response)               
                    }       

		    if (error){
	    		throw ( UnexpectedError )
		    }else{
			res << response
         	}                  
            }
        }]
    }
}
\end{lstlisting}
\end{listing}

Our implementation of the circuit breaker embedder ~\ref{circuitbreaker-client.ol} requires defining an operation \textit{callback}, which receives a message from the circuit breaker and redirect the request to proper operation. In this case, the workflow includes checking the request operation from \textit{operation} field then forward the message to an internal input port, defined using \textit{local} scheme for location. After the call is complete, return the corresponding result to the circuit breaker.

We had demonstrated an implementation of the microservice architecture, the circuit breaker, using the new feature of the module system. The Jolie module system enables Jolie developer to create and compose multiple sole responsibility services and create a more complex application using the import mechanism.


