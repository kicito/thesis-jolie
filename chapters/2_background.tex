\chapter{Background}

\section{Jolie Programming Language}

\subsubsection{Type and Value Representation}
\label{sec:jolie-type-def}

The type system in Jolie is based on structural typing\cite{jolie-type-2019}, where the compatibility and equivalence are determined from the actual structure of the comparing target. It is used to define the message's structure both incoming and outgoing. A type can be defined as a nested tree where each node contains a value. Jolie supports common primitive types with an additional two build-in types \textbf{raw} and \textbf{any}. The \textbf{raw} type is an internal type used for passing data, and the programmer cannot create it. The \textbf{any} type indicates that the value can be in any form of the basic types. One of Jolie's data structure's key features is that every variable is a dynamic array.

The type cardinality can be defined to ensure the range of occurrences for a node in the form of \textit{[ min, max ]} where min and max is a positive number that indicates the range. The common quantifier shorthands such as ? and *  are also supported here. If the cardinality is not defined, it will use the default value of [1,1]. As shown in the reference of \texttt{pet} field in figure ~\ref{list:TypeDefinitonUsage}.

The syntax of Jolie type definition grammar is shown in figure ~\ref{fig:TypeDefinitonSyntax}.

A type can be defined in three ways:
\begin{itemize}
	\item \texttt{typeInline} is a structural composition of types, as illustrated in figure ~\ref{list:TypeDefinitonUsage}.
	\item \texttt{typeLink} is a definition aliasing to a previously defined type.
	\item \texttt{typeChoice} is a type that can be represent from multiple types.
\end{itemize}

\begin{figure}
	\begin{framed}
		\begin{grammar}
			<typeDefinition> ::= `type' <ID> `:' <type>

			<type> ::= <typeInline>
			\alt <typeLink>
			\alt <typeChoices>

			<typeChoices> ::= <type> `|' <type>

			<typeLink> ::= <ID>

			<typeInline> ::= <nativeType> <subTypeBlock>?

			<subTypeBlock> ::= `{' `.'? <ID> <cardinality>? `:' <type>? `}'

			<cardinality> ::= `[' int `,' ( int | `*' ) `]'
			\alt `*'
			\alt `?'

			<nativeType> ::= 'void' | 'string' | 'raw' | 'any' | 'int' | 'double' | 'long' | 'bool'
		\end{grammar}
	\end{framed}
	\caption{Grammar for Jolie's type Definition \protect\footnotemark}
	\label{fig:TypeDefinitonSyntax}
\end{figure}

\footnotetext{ID denote an identifier}

\begin{listing}
	\begin{sublisting}{\linewidth}

		\lstset{language=Jolie,
			style=codeStyle
		}
		\begin{lstlisting}[frame=tlrb]{typeDef-Jolie}
type person: int{
	name: void {
		firstname: string
		lastname: string
	}
	isMarried: bool
	pet[0, 2]: string
	age: int
}

person = 1
person.name.firstname = "John"
person.name.lastname = "Doe"
person.isMarried = false
person.pet = "Johnny"
person.pet[1] = "Max"
person.age = 20
\end{lstlisting}
		\caption{ Construction of type and value in Jolie }
		\label{list:type-value}
	\end{sublisting}\\[2ex]
	\begin{sublisting}{\linewidth}
		\lstset{language=XML,
			showspaces=false
		}
		\begin{lstlisting}[frame=tlrb]
<person>
	1
	<isMarried>false</isMarried>
	<name>
		<firstname>John</firstname>
		<lastname>Doe</lastname>
	</name>
	<pet>Johnny</pet>
	<pet>Max</pet>
	<age>20</age>
</person>
\end{lstlisting}
		\caption{ XML representation of \ref{list:type-value} }
		\label{list:type-value-xml}
	\end{sublisting}
	\caption{ Jolie type construction and its representation }
	\label{list:TypeDefinitonUsage}
\end{listing}

\FloatBarrier

\section{Module System}

Module system is a system consisting of composition of an independent executable program unit. A program unit contains every definition and functionality intended to solve one specific problem, each program unit is called Modules in the module system. Each module can be parsed into a module interface, which defines a name of public symbols and the required symbols from other modules. The module interface allows other modules access providing symbols and use its implementation details in their environment. The Module System targets the decomposition of large applications into smaller pieces. This led the developer of the supported module system programming language to reuse their codes and allow the concept of package managing tools, which helps engage its community to collaborate more and grow faster.

Usage of terms in the Module system varies between programming languages as their implementation detail of the module system. Before diving into the module's system of each selected popular programming language, the reader will be introduced with the term that is used within their context.

Import statement is the statement that is commonly used in the Modular programming language to access and retrieve definitions between modules. It is generally responsible for two actions. Firstly locate the target module and resolve the definition to the local execution environment.

In the sections below, we will look at popular programming languages that are famous for its module system. We focus on the behavior of their design choice of import system and import statement syntax.

\subsection{Python Module System}

The Python module system was inspired by Modula-3, which introduces a definition of modules and that is also used in Python.  Python also acquired the Modula-3 import statement syntax so-called dotted-separate name to allow traversing between items within a module target.  Python’s import system behaviour has been gradually evolving. Here in this chapter, we will discuss how Python 3.0 defined their import system and their import statements behaviour.

In order to understand a programming language’s import system, the definitions of packages, modules, and symbols have to be defined first, these definitions are retrieved from official Python 3 documentation

A module in python is an object with a list of python definitions and executable statements. These statements are executed during the initialization of the module. In the file system perspective a module is file with py extension, and module is named after the file name.

A package is also a module which might contain submodules or a subpackages, in an execution environment packages is a module with an \_\_path\_\_ attribute. Python packages are directories in the file system. In the current version of Python, the packages are categorized into two types called regular packages and namespace packages. A regular package is a package that is used in Python pior version 3.2. It’s implementation is a directory containing \_\_init\_\_.py file. A namespace package, defined as directory in a file system without \_\_init\_\_.py file. It is a combination of python’s specific element called portions. Which is a set of files in a single directory or an entry in a zip file. Portions in a namespace package may reside in different locations in the python search location. Namespace package is mainly designed to support an ability to separate submodules in multiple locations as long as the portion is found by the module finder.

A regular package is a directory containing \_\_init\_\_.py file. This file is executed when the package of this type is imported, and any object defined in \_\_init\_\_.py will be bound to package namespace. The \_\_init\_\_.py file is also used by package developer to control definitions to be importwhen the local importer specify a wildcard character(*) as an identifier target; this is done by assigning exported symbols to \_\_all\_\_  symbol.

Despite various ways for a module to invoke an import mechanism, we mainly focus on an import statement of Python. The location of the target module can be specified by a dotted separate module name, while the target symbols is used to indicate symbol name in local namespace to be referred later for import statement caller or a wildcard character. The syntax of import statements in Python is defined as follows.


% \begin{lstlisting}[style=ANTLR ,float, caption={Import statement syntax for python 3},captionpos=b]
% import_stmt: import_name | import_from;
% import_name: 'import' dotted_as_names;
% // note below: the ('.' | '...') is necessary because '...' is tokenized as ELLIPSIS
% import_from: ('from' (('.' | '...')* dotted_name | ('.' | '...')+)
%               'import' ('*' | '(' import_as_names ')' | import_as_names));
% import_as_name: NAME ('as' NAME)?;
% dotted_as_name: dotted_name ('as' NAME)?;
% import_as_names: import_as_name (',' import_as_name)* (',')?;
% dotted_as_names: dotted_as_name (',' dotted_as_name)*;
% dotted_name: NAME ('.' NAME)*;
% \end{lstlisting}

An import statement of python has two forms, it is either in form of import ... and from … import ... statements. Both statements create identical instructions to the module importer, choice of use is subjectively open to the developer to pick. The syntax of these to form is differ from each other in the following way.

The first form of import, import item.subitem.subsubitem, each item except the last element must be a package, while the last item can only be either a module or a package; the definition symbol cannot be the last element in this form.

For the second form, from package import item, the item element can be a submodule or any definition symbol declared within the package. In this form import statement, it performs a lookup to symbol item inside the package, if not found, the process continues with an attempt to import module item. The table below shows the translation of symbols identified in different form of import symbol


ImportStatement
Identifier detail
import item
item can either be a package or module
import item.subitem
item is a package while subitem is a submodule in item
from pkg import item, item2



If pkg is a module, item and item2 are definition symbols in pkg
If pkg is a package, item and item2 can either be a submodule in pkg or a definition defined in pkg.
from pkg.mod import item
pkg is a package and mod is a submodule in pkg, while item can be a submodule of mod, or a definition symbol defined in mod
Import statement translation

The import statement, after completing its execution, resulted in a set of defined definitions ready to use in the caller’s execution context. In order to achieve this goal, import statement combines two operations i.e. searching of the named module and binding result of loaded module to a name in local scope. Search operations involve certain side-effects to the execution environment, such as importing the parent packages, and updating a module’s cache object. While the binding result process is simply assignment of the target imported module’s definitions into appropriate symbols.
Python performs an execution of an import statement by an abstraction object called Importer. The importer consists of two abstraction objects; finders and loaders. Finders are responsible for performing module lookup from the import statement. Each finder constructs a fully qualified name of the module target by its strategy and produces necessary information for the loader to load a module afterward. Preset finders in python are one that is able to find built-in modules, frozen modules, and the last one which will perform a lookup at the list of locations defined in the import path. The loader uses the result from finder to execute the module and create a module object  and save it’s result to the cache and binds name according to the calling import statement in the local execution environment.
Python mechanism to locate a module is performed by a finder. A finder can be extended by two types of finders which are based on the main location where the finder should attempt to find the module. Firstly the MetaPathFinder will perform a lookup at a list of paths passing through the finding method. The second type of finder, the PathEntryFinder, performs look up only to the path that is assigned during its creation.


% SoC && processes หลักในการทำงาน
% - Jolie Components ตาม syntax
%    - symbol declaration


%        - TypeDefinition
%        - InterfaceDefinition
%    - behavior declaration
%        - Ports
%        - EmbedService
%    - program example

% Module System ในภาษาต่างๆ
% - Python
