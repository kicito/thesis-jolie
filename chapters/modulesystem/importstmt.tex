
\section{Import Statement}
An import statement allows Jolie module to gain an accessibility to symbols from others and copy the definitions into local execution the statement syntax is shown in figure ~\ref{fig:jolie-import-stmt-syntax}.

\begin{figure}[ht]
    \begin{framed}
        \begin{grammar}
            <importStmt>
            ::= `from' <importingTargetModule> `import' <importingTargetSymbols>

            <importingTargetModule> ::= ( `.' )* ( <dottedName> | `.' )

            <dottedName>
            ::= <NAME> ( `.' <NAME> )*

            <importingTargetSymbols> ::=  `*' | <importAsNames>

            <importAsNames>
            ::= <importAsName> ( `,' <importAsName> )*

            <importAsName>
            ::= <NAME> ( `as' <NAME> )?

            <NAME> ::= <ID>
        \end{grammar}
    \end{framed}
    \caption{Jolie import statement syntax }
    \label{fig:jolie-import-stmt-syntax}
\end{figure}

Import statement consisted of importing module target which is a dotted separated packages name with a module as the last element. Followed by a list of importing symbols target or a wildcard character ( `*' ), which targets every symbols in the importing target.
Jolie import statement is similar to the one that Python has, thus this makes it approachable to programmers who are new to the language.
the main advantage behind the decision of using \texttt{from \dots import \dots} syntax is the essence of specifying a module target before symbols helps the language server extension on editor to perform better suggestion to symbols.
