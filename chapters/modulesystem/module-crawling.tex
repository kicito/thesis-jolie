\section{ Module Crawling }

Module Crawling is a process of locating and parsing the dependencies of the execution module.
The module crawler is essential for the Jolie module system in such a way that it constructs a set of required module records for the execution of a program.

The crawling process takes an initial module record and scans for the dependent modules. After a dependent module has been found, it is parsed and scanned for its dependent modules. These dependencies are stored in a queue data structure with a cache. The module crawler repeatedly performs this procedure until every dependent module has successfully parsed into the Symbol Table. The algorithm for this function is defined in procedure \texttt{crawl} in algorithm ~\ref{algo:crawl}.

\begin{algorithm}
    \begin{algorithmic}[1]

        \Require{A crawling module record $root$}
        \Ensure {Module records required for the execution $result$}

        \Procedure {Crawl}{$root$}

        \State $Q\gets \Call{FetchDependency}{root}$ \Comment{module to crawl}
        \State \Call{$result.put$}{$root$} \Comment{add root to result}

        \While{$Q$ is not empty}
        \State $module\gets \Call{Q.poll}{}$

        \If{$module$ in $result$}
        \textbf{continue} \Comment{record is already parsed}
        \EndIf

        \State $record\gets$ \Call{parse}{$module$} \Comment{parse source into a module record}

        \State $deps\gets \Call{FetchDependency}{record}$
        \ForAll {$d$ in $deps$}
        \If{$d$ not in $Q$}
        \State \Call{$Q.add$}{$d$}
        \EndIf
        \EndFor
        \State \Call{$result.put$}{$record$} \Comment{add record to result}

        \EndWhile
        \EndProcedure

        \Function {FetchDependency}{$mr$}
        \State $deps\gets$ new list
        \ForAll{$s$ in $mr$'s external scope symbols}
        \State $moduleTarget\gets$ module target of $s$
        \State $source\gets \Call{find}{moduleTarget}$ \Comment{find operation by finder}

        \State \Call{$deps.insert$}{$source$}
        \EndFor

        \State \textbf{return} $deps$ \Comment{list of dependency module for $mr$}
        \EndFunction

    \end{algorithmic}
    \caption{Jolie module system's module crawling algorithm}
    \label{algo:crawl}
\end{algorithm}
\FloatBarrier

The Module finding is performed by a component of the module crawler called finder. The current implementation of Jolie module system has two types of finders, which define different strategies to locate a module. The Absolute Finder performs a lookup at a set of defined path locations. The Relative Finder performs a search from a relative path to the import statement caller. The finder is responsible for finding an imported module and making sure that the import target exists in the file system. If the finder failed to retrieve the module's information, the interpreter raises a ModuleNotFound error and stops its execution.

\subsection{ Absolute Path Resolution }

The absolute import, which performs a search for the module at a defined path location, is used for importing a library that does not reside in the package-specific modules. The use case for such an import resolution strategy is that of built-in modules.

The Absolute Path Finder performs a lookup in two different locations. The working directory of the execution process is the first place to perform a lookup, where the finder directly tries to resolve the target module. Then, it continues to the directory \texttt{lib} to look for \texttt{first.jap} where \texttt{first} is a first token defined in module target. Lastly, the finder will perform module lookup to the extensible list of directories defined in a special path called \texttt{packLib}. It contains a path to built-in libraries and can be extended through command-line arguments. The procedure for the absolute path resolution is shown in algorithm ~\ref{algo:absolutePath}:

\begin{algorithm}[H]
    \begin{algorithmic}[1]
        \Require{A module target $target$}
        \Require{A working directory of execution process $wdir$}
        \Require{A list of package directories $libs$}

        \Ensure {A location to the existed module in file system}

        \Procedure {find}{}
        \State $P\gets$ \Call{$target.split$}{`.'} \Comment{package/module tokens}
        \State $first\gets P[0]$ \Comment{first element of $P$}
        \State $rest\gets P[1...]$ \Comment{elements after first of $P$}

        \State $module \gets$\Call{resolve}{$wdir$, $P$}
        \If{$module$ existed}
        \State \textbf{return} $module$ \Comment{target module located in $wdir$}
        \EndIf

        \State $japPack \gets$\Call{resolve}{$wdir$, `lib', $first$}
        \If{$japPack$ existed and has jap extension}
        \If{$japPack$ has $rest$ as a entry}
        \State \textbf{return} $japPack$
        \EndIf
        \EndIf

        \ForAll{$lib$ in $libs$}
        \State $module \gets$\Call{resolve}{$lib$, $P$}
        \If{$module$ existed}
        \State \textbf{return} $module$ \Comment{target module located in $lib$}
        \EndIf
        \EndFor

        \State \textbf{return} None \Comment{lookup failed}

        \EndProcedure

    \end{algorithmic}
    \caption{Jolie module system's module finding algorithm for an absolute import}
    \label{algo:absolutePath}
\end{algorithm}

\begin{figure}[ht]
    \begin{forest}
        for tree={
        font=\ttfamily,
        grow'=0,
        child anchor=west,
        parent anchor=south,
        anchor=west,
        calign=first,
        edge path={
                \noexpand\path [draw, \forestoption{edge}]
                (!u.south west) +(7.5pt,0) |- node[fill,inner sep=1.25pt] {} (.child anchor)\forestoption{edge label};
            },
        before typesetting nodes={
                if n=1
                    {insert before={[,phantom]}}
                    {}
            },
        fit=band,
        before computing xy={l=15pt},
        }
        [.
            [lib
                    [someLib.jap
                            [module.ol]
                    ]
            ]
            [packages
                    [somePackage.ol]
            ]
        main.ol
        ]
    \end{forest}

    \lstset{language=Jolie,
        style=codeStyle,
        numbers=left,
        firstnumber=1
    }
    \begin{lstlisting}[frame=tlrb]{import-statement-absolute.ol}
from someLib.module import componentA
from packages.somePackage import componentB
from console import consoleService
\end{lstlisting}
    \caption{Demonstration of absolute import usage in Jolie's module system}
    \label{fig:jolie-absolute-import}
\end{figure}
Consider figure ~\ref{fig:jolie-absolute-import}, which shows a working-directory tree representation and a working-code snippet in \texttt{main.ol}, demonstrates the relation between module targets and the actual working directory structure. The import statement defined at the first line demonstrates an example of importing a jap archive file, while the second line is resolved into a ./packages/somePackage.ol. Lastly, the last line is an example of importing a built-in module \texttt{console}.


\FloatBarrier
