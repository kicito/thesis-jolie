
\section{ Name Resolution }

The introduction of the Symbol Table allows the Jolie interpreter to parse the definition referenced in the abstract syntax node by using a lazy evaluation strategy. This benefits Jolie developers by allowing symbol referencing without worry of the definition order. To illustrate the problem Jolie had before, consider the following source code:

\begin{listing}[H]
    \lstset{language=Jolie,
        style=codeStyle
    }
    \begin{lstlisting}[frame=tlrb]{definitionOrder.ol}
define bar{
    foo
}

define foo {
    bar
}
\end{lstlisting}
\end{listing}

The interpreter will fail to parse this source code since the parser cannot find a procedure definition \texttt{foo} when attempting to parse the token \texttt{foo} inside the procedure \texttt{bar}.

By introducing name resolution to the current workflow, it is possible to write and use definitions without a concern on order. During the parsing phase, the Jolie interpreter assigns an empty named symbol object, which will be resolved to an expected abstract syntax node by performing name resolution.