
\section{Symbol and Symbol Table}

The symbol table is a well-known data structure in programming languages. Generally, it is used as a truth table of reference between a named declaration, or symbol, associated with the source code. The term symbol refers to unique entities defined and used in the execution environment; the defined entities are such as variable names, function names, classes, and so on.
The Symbol Table is responsible for maintaining a data structure of this information and its description, such as the occurrences, declaration scope, and type. This data structure is useful throughout the program execution lifecycle, such as the analysis tasks and the optimization.

For the module system in Jolie, the Symbol Table is implemented with a hash table and associated with a single module.
The symbol contains essential information for performing binding in later stage this include: symbol name, scope, access modifier, and a pointer to abstract syntax node.
Scope indicates the location of the declaration node of the particular symbol, either internal or external; internal symbol specifies that the symbol definition is declared within the module; on the contrary, external symbol, created from an import statement, determines that the interpretation is defined from the external source.
An access modifier defines the accessibility of a symbol from the external module. It can be defined as \textbf{public}, which indicates that the symbol is accessible from external modules and allows to be imported. On the contrary, \textbf{private} prohibits access from the external module. A symbol is set to \textbf{public} by default; it can be defined as private by annotate a \textbf{private} keyword before the declaration of a symbol.

As the introduction of the symbol table, we have the definition of the abstract syntax node as defined in figure ~\ref{sec:jolie-def} and, additionally, the Service node, implemented as a newly added symbol. The Service node is a new abstract syntax node of Jolie for declaring a service, which we will get back to in later sections.

The construction of the symbol target consists of two stages. During the first stage, definition nodes are visited and pushed as a new entry in the symbol table. After this stage, every symbol defined within the module is bound to an abstraction node. In the second stage, a simple name lookup assigns the entry of an external module record, the corresponding abstract syntax node to the referring module's symbol table.

\begin{figure}[ht]
    \begin{subfigure}[b]{\textwidth}
        \lstset{language=Jolie,
            style=codeStyle,
            numbers=left,
            firstnumber=1
        }
        \begin{lstlisting}[frame=tlrb]{symboltable.ol}
from C import *
from .B import b_type as b_imported
type A {
    a_type: int
}
private type from_b: b_imported
\end{lstlisting}
    \end{subfigure}
    \begin{subfigure}[b]{\textwidth}
        \begin{tabular}{ |c|l|l|l| }
            \hline
            Symbol name & scope        & privacy & binding abstract syntax \\
            \hline
            C*          & external(C)  & private & NONE                    \\
            b_imported  & external(.B) & private & NONE                    \\
            A           & local        & public  & $<$ type: A $>$         \\
            from_b      & local        & private & $<$ type: from_b $>$    \\
            \hline
        \end{tabular}
    \end{subfigure}
    \caption{Jolie import statements and representation of its symbol table}
    \label{fig:jolie-ex-symbol-table}
\end{figure}

From figure ~\ref{fig:jolie-ex-symbol-table}, which demonstrates the construction of Symbol table, demonstrates three entries that get created at the first step of the symbol table creation. The detail of each symbol is the following:

\begin{itemize}
    \item \texttt{C*}, originated from an import statement, is instantiated as an \texttt{external(C*)}, which determines that the definition node is referring to an external source module named B. In this first step, the symbol name is assigned as a unique name.
    \item \texttt{b_import}, originated from an import statement, is instantiated as an \texttt{external(.B)} which determines that the definition node is referring to an external module named B at a relative location with respect to the execution node.
    \item \texttt{A}, defined within the local execution is a local scope symbol with the \textbf{public} as the default privacy.
    \item \texttt{from_b}, has the keyword \textbf{private} before the declaration; thus, this symbol cannot be imported by others.
\end{itemize}

After this step, the partially created symbol table of the module is created. This partially filled symbol table has information on the syntax node defined within its module. The external-scope symbols are left without pointers to their binding abstract syntax node. To fully fill this field, we can simply perform a lookup on the symbol table of the module target that the declaration is originated from.

Consider figure ~\ref{fig:jolie-ex-symbol-table-completed} which defines a symbol of module \texttt{C} and the completed symbol table from previous example. Entry \texttt{C*} is replaced by the exposed symbol in its module.

\begin{figure}[ht]
    \begin{subfigure}[b]{\textwidth}
        \begin{tabular}{ |c|l|l|l|l| }
            \hline
            Source              & Symbol name & scope        & privacy & binding abstract syntax        \\
            \hline
            \multirow{3}{*}{C}  & C1          & local        & public  & $<$ type: C1 $>$               \\
                                & C2          & local        & private & $<$ type: C2 $>$               \\
                                & CIface      & local        & public  & $<$ interface: CIface $>$      \\
            \hline
            \multirow{4}{*}{.B} & C1          & external(C)  & private & $<$ type: C1 $>$ in C          \\
                                & CIface      & external(C)  & private & $<$ interface: CIface $>$ in C \\
                                & b_imported  & external(.B) & private & $<$ type: b_type $>$ in B      \\
                                & A           & local        & public  & $<$ type: A $>$                \\
                                & from_b      & local        & private & $<$ type: from_b $>$           \\
            \hline
        \end{tabular}
    \end{subfigure}
    \caption{Completed representation of a symbol table in figure ~\ref{fig:jolie-ex-symbol-table}}
    \label{fig:jolie-ex-symbol-table-completed}
\end{figure}

\subsubsection*{Module Record}

From here, we can define a new abstraction object in the Jolie interpreter ecosystem, considering that the symbol table is a representation of symbol usages in a module, and a module is parsed into a single abstract syntax tree node. The Module Record is an object consisted of an AST and symbol table of a module it's referred to.
